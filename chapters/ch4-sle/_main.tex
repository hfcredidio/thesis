\chapter{Schramm-Loewner Evolutions}
\label{ch:sle}

\section{Numerical Methods}
\label{sec:numsle}

Analyzing numerically SLEs to solving Lowener's equation

\begin{equation}
    \partial_t g_t = \frac{2}{g_t - U_t}
\end{equation}

The main objective is to find the trace $\gamma_t = g_t^{-1}(U_t)$. The most
straightforward way to do this is to use a simple Euler Scheme. In first order
it consists in iterating the following formula
\begin{equation}
    g_{t+\Delta t} = g_t + \Delta t \frac{2}{g_t - U_t}
\end{equation}

As shown in Eq. XX, as time passes, the imaginary part of the function $g_t(z)$
approaches zeros, that is, it approaches the real line. Therefore we can color
the sign of the real part of the map, getting an image like the one show in
Fig. XX. The border between the regions is the resulting SLE trace of the
process. We observe the line presents the properties. The smooth tail observed
happens because there was not enough time to the points in the vicinity to be
mapped to the real line.

This methods have the advantages of being very simple to program and fairly
quick to run. However this method behaves badly enough that for large $\kappa$ 
one cannot obtain a discernible trace. Even a bad one.

Another, much better method is the zipper algorithm. The main idea behind it is
breaking the SLE process into shorter processes during which the driving function
takes a smooth form. 

We start by taking a set of $N$ discretized timesteps $0=t_0 < t_1 < t_2 <
\ldots < t_N$ where the driving function takes the value of $U_{t_i}$.

The discretizing function can be chosen at will, but it's desirable that
it have an analytical solution. Two common choices are the vertical slit and
the tilted slit. The first one considers that for $t\in[t_i, t_{i+1}]$ the
driving function take the (in other words, it is a zero order interpolation).
The vertical slit makes it have the form

