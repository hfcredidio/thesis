\chapter{Introduction}
\label{ch1-intr}

Lorem ipsum dolor sit amet, consectetur adipisicing elit, sed do eiusmod tempor
incididunt ut labore et dolore magna aliqua. Ut enim ad minim veniam, quis
nostrud exercitation ullamco laboris nisi ut aliquip ex ea commodo consequat.
Duis aute irure dolor in reprehenderit in voluptate velit esse cillum dolore eu
fugiat nulla pariatur. Excepteur sint occaecat cupidatat non proident, sunt in
culpa qui officia deserunt mollit anim id est laborum.

%Sources:
    %Nishimoto - preface
    %Christensen - preface
    %Sole - preface
    %Cardy - intro

%Mr. Broadbent used to live in a nice rustic cabin by a lake. One day, during a
%walk through his property he found a single little frog. This was unusual, Mr.
%Broadbent knew that there was a frog breeding spot on the other side of the
%lake, but he had never seen a frog on this side. This little one must have
%gotten lost. It's worth mentioning that Mr. Broadbent was not a big fan of
%frogs, one might say he's terrified of them, but he has a heart after all and
%wanted to help this little fellow to get back to its family and friends,
%although he couldn't suffer to take it the other side himself. Luckily he had a
%sizable stock of lily pad seeds. He could throw some at the lake, and with
%enough luck the lily pads that grow would form a bridge that the frog would be
%able to cross. Mr. Broadbent knew that wherever he throws the seeds, they might
%grow anywhere in the lake because of some water currents that happened there.
%But how many lily pads should he plant? If he throws too few seeds in the lake,
%the likelihood that a lily pad bridge will form is very small. This could be
%solved be simply throwing enough seeds to cover the whole surface area of the
%lake (as said, he has a \textit{lot} of lily pad seeds, something to do with an
%inheritance from an eccentric uncle), but this would ruin the delicate
%ecosystem of the lake, not to mention significantly lower the value of his
%property. Mr. Broadbent needed to know what is the smallest number of seeds, or
%in other words, what's the smallest area of the lake that he needed to cover so
%that he's sure to form a lily pad bridge for the lonely frog to cross.

%Mr. Broadbent so happens to be a computational physicist and can easily write a
%computer program to compute the smallest area needed to cover the lake in order
%to form a bridge that the frog can cross. He uses the information that each
%lily pad can have a maximum of 6 neighbors, since they're pretty circular and
%have basically the same size. After some days running calculations (his
%computer is really old), he finally comes to an answer. Mr. Broadbent needs to
%cover exactly 50\% of the lake surface. Below that point he can be almost sure
%there will not be a bridge formed, and above that he can be certain that a
%bridge will be there.

%Mr. Broadbent wasted no time and threw enough seeds to cover half of the lake
%area and lo and behold, in the following days a bridge was formed and his little
%green friend crossed it going back to his homeland. That night Mr. Broadbent
%went to bed feeling proud, he helped another living being, and he used math and
%computers to do so! To his terror however, he woke up the next day to an
%infestation of frogs in his yard. It happens that if one frog can cross the
%bridge one way, many others can cross it the other way around. I didn't took
%too long for the frogs to make another breeding ground on this side of the lake
%making the infestation even worse.

%One might think Mr. Broadbent was desperate, but he is a scientist, and more
%than anything he is intrigued. There was always lily pads in the lake, and they
%could keep growing and as long as the fraction of area they cover remained
%below 50\%, nothing would have changed, because no bridge between the sides
%would have been formed. But as soon as the lily pads cover more than half of
%the lake surface area all hell broke loose, you have frogs everywhere!
%Strangely nothing actually changed, lily pads are still lily pads, the frogs
%are still the same frogs and the lake is still the same lake, but the overall
%``behavior'' of the lake changes drastically, it goes from an essentially
%frogless states to a froful one.
%Mr. Broadbent is not an uninitiated physicist, he know what he just witnessed.
%He have observed first hand a \textit{phase transition}.

%Phase transition have been for more than a century a constant object of
%study in statistical physics.

%The first work that presented what could be considered a modern view of phase
%transitions was done by Charles Cagniard de la Tour in 1838 [..] where he
%reported a unusual phenomenon, that when subject to a substance specific value
%of pressure and temperature, several compounds started showing a milky aspect,
%which means that light was being strongly scattered. Above this point, liquid
%and gaseous phases of matter become indistinguishable. In 1869 Thomas Andrews
%measured to high precision the point at which several substances showed this
%phenomenon, coining the term \textit{critical point} as the point in phase
%space where this \textit{critical opalescence} is observed.

%On of the most astonishing aspect of critical phenomena was first observed by
%Pierre Curie in 1895[..], who noticed the similarities between the liquid-gas
%phase transition discovered by de la Tour and the ferromagnetic transition
%(where a metal lose its ferromagnetic properties beyond some critical
%temperature). This universal behavior, that is independent of the detail of the
%systems was only started to be well understood in the 60's and 70's with 
%the works of...
