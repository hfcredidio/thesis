\chapter{Introduction}
\label{ch1-intr}

%Sources:
    %Nishimoto - preface
    %Christensen - preface
    %Sole - preface
    %Cardy - intro

The existence of phase transitions have been a part of the human experience ever
since a person first boiled a pot of water, or saw the melting of snow. The
idea that a substance can radically change its physical properties without
changing its composition raises a number of questions that have only started to
be answered in the last century or so. The quest for understanding what
drive(s?) these changes on a microscopic level, and how it relates to the
peculiar macroscopic behavior we observe is the starting point of the complex
but profoundly insightful area of phase transitions and critical phenomena (two
terms that, albeit not strictly the same, are commonly used interchangeably).

Although the phenomenon of phase transitions is known since immemorial time,
the first scientific observation that can be linked to the modern theory of
critical phenomena was made by Cagniard de La Tour in 1822~\cite{delaTour1822},
who observed that under certain conditions of temperature and pressure the
surface tension between the liquid and gas phases of several substances
disappears, thus discovering what we call today a supercritical fluid phase. In
1869, Thomas Andrews coined the term \textit{critical point} for the point in
phase space where the liquid and gas phases become indistinguishable
\cite{Andrews1869}. He also noted the milky opaque aspect the substances took
when near the critical point, which he called critical opalescence.
In 1895, Pierre Curie discovered the ferromagnetic transition, when a material
lose its magnetic properties when subject to a high enough temperature
\cite{Curie1895}. More importantly, he was the first to notice universal
characteristics in the critical behavior of different systems. This property
would eventually be regarded one of the pillars of critical phenomena
theory~\cite{Stanley1999}.

At the same time as these discoveries were being made, a new area of physics
was flourishing: statistical mechanics. After Gibb's foundational work
\cite{Gibbs1906} it was pretty clear that this new formalism was the perfect
fit for exploring critical phenomena, which allowed the first half of the 20th
century to bear witness to great advances including the introduction of the
ubiquitous Ising model in 1925~\cite{Ising1925} and its celebrated solution in
2D by Onsager in 1944~\cite{Onsager1944}. The 50's saw the introduction of the
elegant and broadly applicable percolation theory by Broadbent and
Hammersley~\cite{Broadbent1957}.

What is considered the modern era of phase transitions started in the 60's when
several theoreticians (including Heller, Benek, Jacrot, Domb, and many others)
came to the realization that the critical point exponents were entities that
deserved special attention~\cite{Stanley1999, Stanley1971}. This was followed
by another major advance in the form of renormalization group theory, led by
Leo Kadanoff~\cite{Kadanoff1966}.

The 80's saw the application of conformal invariance as an extension of scaling
invariance, which, along with with a well developed field theoretical
apparatus, allowed for the computation of the critical exponents of various
models~\cite{Belavin1984, Henkel2013}. Along that, propelled by the idea of
self-organized criticality~\cite{Bak1987}, the scope of applications for the
concepts of scale invariance and criticality started expanding well beyond the
physics realm, reaching areas as distinct as ecology, economics and social
sciences~\cite{Bak1996, Christensen2005}.

The last big theoretical step in phase transitions came from mathematicians
unsatisfied with the hodge-podge of implicit assumptions and unproved
statements that make a lot of field theory~\cite{Langlands1994, Cardy2005}.
Using techniques from complex analysis developed in the 20's by Charles
Loewner~\cite{Loewner1923}, Oded Schramm came up with an elegant formalism he
called stochastic Loewner evolutions~\cite{Schramm2000}, but appropriately
renamed Schramm-Loewner evolutions after its resounding success in explaining
the critical properties of a number of phase transition models, most notably
percolation. Not surprisingly the area yielded two Fields
medals~\cite{Mackenzie2006, Kesten2010}. Sadly none of these were awarded to
Schramm himself, because he was ineligible for being over 40 years old (by less
than a month!).

The study of phase transitions and critical phenomena is a century-old theory
in the making. We endeavor to take a small step in its advancement by exploring
the critical properties of an often ignored class of systems, the ones that
present strong anisotropy~\cite{Henkel1994}. We will do that under the umbrella
of Schramm-Loewner evolution formalism, and, with help of numerical
simulations, shed some light in the relationship between anomalous diffusion
and anisotropic scaling.
