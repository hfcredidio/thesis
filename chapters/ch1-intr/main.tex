\chapter{Introduction}
\label{ch1-intr}

%Sources:
    %Nishimoto - preface
    %Christensen - preface
    %Sole - preface
    %Cardy - intro

The idea that a substance can radically change its physical properties without
changing its composition has always captivated the curious mind of scientist
throughout the centuries.
Phase transitions and critical phenomena (two terms that, albeit not strictly
the same, are commonly used interchangeably).

Although the phenomenon of phase transitions is known since immemorial time,
the first scientific observation that can be linked to the modern theory of
critical phenomena was made by Cagniard de La Tour in 1822~\cite{delaTour1822},
who observed that under certain conditions of temperature and pressure the
surface tension between the liquid and gas phases of several substances
disappears, thus discovering what we call today a supercritical fluid phase. In
1869, Thomas Andrews coined the term \textit{critical point} for the point in
phase space where the liquid and gas phases become indistinguishable
\cite{Andrews1869}. He also noted the

Pierre Curie discovered the ferromagnetic transition in 1895, when a material
lose its magnetic properties when subject to a high enough temperature
\cite{Curie1895}. More importantly, he was the first to notice universal
characteristics in the critical behavior of different systems. This property
would eventually be considered one of the pillars of critical phenomena
theory~\cite{Stanley1999}.

At the same time as these discoveries were being made, a new area of physics
was flourishing: statistical mechanics. After Gibb's foundational work
\cite{Gibbs1906} it was pretty clear that this new formalism was the perfect
fit for exploring critical phenomena. The first half of the 20th century saw
great advances including the introduction of the ubiquitous Ising model in 1925
\cite{Ising1925} and its celebrated solution in 2D by Onsager in 1944
\cite{Onsager1944}. The 50's saw the introduction of percolation theory by
Broadbent and Hammersley~\cite{Broadbent1957}.

What is considered the modern era of phase transitions started in the 60's when
several theoreticians (including Heller, Benek, Jacrot, Domb, and many others)
came to the realization that the critical point exponents were entities that
deserved special attention~\cite{Stanley1999, Stanley1971}. This was followed
by another major advance in the form of renormalization group theory, led by
Leo Kadanoff~\cite{Kadanoff1966}.

The 80's saw the application of conformal invariance as an extension of scaling
invariance, which, along with with a well developed field theoretical
apparatus, allowed for the computation of the critical exponents of various
models.

In the last two decades the scope of applications for the concepts of scale
invariance and criticality extended well beyond the physics realm, reaching
areas as distinct as biology, economics and social sciences.

The last big theoretical step in phase transitions is SLE\@.

[We want to extend a liitle]  this century-old theory in the making.
