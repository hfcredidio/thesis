\chapter{Strongly Anisotropic Systems}
\label{ch:anis}

So far we introduced the important notion of scale invariance and how it can be
used to better understand critical systems. This kind of symmetry emerges in
static (as in no temporal dependences in the thermodynamical quantities) and
isotropic.

Compared to isotropic scaling.

\begin{equation}
    C(a\vec{r}, a^\theta t) = a^{2x} C(\vec{r}, t)
\end{equation}
where $t$ can stand for any preferential direction of the
system, not necessarily time. If $\theta \neq 1$.

As conformal invariance is a generalization of scale invariance it that
it supposes that systems are scale invariant only locally, a subset
of strongly anisotropic systems have been show to have a similar properties.
These systems are calle Schrodinger invariant systems, and are characterized by
having dynamic exponent $\theta = 2$. Some models have been shown to obbey
this paradigm, but most of them do not display phase transitions in 2D.


\section{Directed Percolation}
\label{sec:dp}

Directed percolation is one of the many variants of the percolation model
described earlier. In general, percolation can be seen as a spreading process,
in which each site is blocked with probability $p$. Bellow the percolation
threshold the systems is macroscopically permeable, that is, any flow of
fluid can run from one side of the lattice to another. 

In the context of strong anisotropy the most important universality class is
that of \textit{Directed Percolation}, in which water can only flow along a
preferred direction.

The most important class of absorbing phase transition. By that we mean phase
transitions that are allowed to enter states that they cannot leave. These
states are called absorbing states. 

Directed percolation is best described as a spreading process. In it, a lattice
row starts with a given initial condition of occupied and unoccupied sites. At
the next time step, each occupied site can spread with probability $p$ to any
of it's neighbors \textit{along a predetermined direction}. One can easily
observe that at any given time the system can enter a completely unoccupied
state, from which it cannot leave.


\section{Multi-Layered Percolation}
\label{sec:mlp}

Another variation of the percolation model that displays strong anisotropy
is the \textit{multi-layered percolation}, introduced by Dayan et Al. [..].
In it, the ...

This model preserves a lot of important properties of isotropic percolation,
most notably the locality property. This property is easily observed by the
fact that the border line of the percolating cluster depends only on the
sites that immediately neighbors the border itself. Because of this, the
same algorithm described in Section [..] can be used to simulate multi-layered
percolation with minimal changes to account for the layered structure.
