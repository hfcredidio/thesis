\chapter{Phase Transitions and Critical Phenomena}
\label{ch:crit}

When trying to analyze a physical system, a big part of a physicist effort is
dedicated to identifying the parts that compose the system and how they
interact with each other. It is usually a daunting task to find an interaction
model that captures all the expected properties of a system. That's because the
macroscopic (i.e.\ collective) behavior does not follow trivially from the
microscopic forces in place. In no area of physics this is more clear than in
that of phase transitions and critical phenomena. Not surprisingly the
relationship between criticality and complexity has been applied to areas well
beyond the domain of atoms and molecules, like cognitive
sciences~\cite{Kello2010}, social scieces~\cite{Kron2009} and
ecology~\cite{Sole1999}, to name just a few.

But let's take a step back and look at the components themselves. We can get
away with making very little assumptions about their actual identities. They
can be elementary particles when studying high energy physics, atoms and
molecules when determining the electronic properties of a material, living
cells when observing the growing pattern of a colony of bacteria, or even
people when trying to design buildings with safe exit routes. What is more
important to us is the fact that these building blocks can organize themselves
in several different ways. Each pattern of organization is called a
\textit{phase}, and in which phase the systems finds itself depends on the
value of some external parameters, like energy density, temperature, nutrient
concentration or whether or not there's a fire in the building. Not only that
but different phases of a same system display wildly different macroscopic
behavior, despite no change taking place in their composition.

The most familiar example is water. At sea level, water is in its solid phase
for temperatures lower than $0^\circ$C. That happens because the water
molecules are locked in place by the cohesive forces that its neighbors exert
on each other. When the temperature is higher than $0^\circ$C but lower than
$100^\circ$C, the molecules acquire enough kinetic energy to be able to move
through one another. They still interact strongly enough to be bound together
but it is not sufficient to keep them fixed. This is the liquid phase. When the
temperature is above $100^\circ$C, water is now in the gaseous state, where
thermal movement dominates and the molecules move mostly unimpeded, interacting
only when colliding directly. See Figure~\ref{fig:phases} for an illustration of
how a molecule moves in the three phases.

Not all phases concern the states of matter, however. Some are related to
magnetic or electric properties. This is the case of ferromagnetic materials,
materials that exhibit a persistent magnetization after being subject to a
strong magnetic field, like your everyday fridge magnet. In contrast,
paramagnetic materials show no permanent magnetization. Ferromagnetism is a
result of the alignment of the magnetic spins of the atoms that compose the
material. If you turn off the external field and raise the temperature,
eventually the thermal fluctuations will misalign the spins, zeroing out the
magnetization and turning a ferromagnetic material into a paramagnetic one.

To better visualize the different phases of systems, it is common to make use
of a \textit{phase diagram}. Simply put, a phase diagram is a graph where each
axis represent one control parameter, and every point in this graph is labeled
according to the phase the system finds itself for that value of the
parameters. Figure~\ref{fig:water}a shows the very famous phase diagram of
water as a function of temperature and pressure. The three main phases (solid,
liquid and gas) are clearly marked, and the black lines are the boundary
between the phases. Crossing any of these lines is followed by a sharp change
on the properties of the system, as described previously. This is know as a
\textit{phase transition}.

\begin{figure}
\begin{center}
    \includegraphics[scale=0.4]{chapters/ch2-crit/figs/phases}
\end{center}
\caption{How a system of particles behave in the three phases of matter. Here a
    simulation of 100 particles was done, but the trajectory of only one is
    shown. In the solid state the particles are confined to small region by
    the interactions of its neighbors. In the liquid state the particle is
    unconfined, but still interacts strongly with the other particles. In the
    gas state the kinetic energy of the particles is large enough that they
    barely interact with one another, making a ballistic trajectory until
    they make a head-on collision with another particle.}
\label{fig:phases}
\end{figure}


\begin{figure}
\begin{center}
    \includegraphics[scale=1.0]{chapters/ch2-crit/figs/water}
\end{center}
\caption{Phase diagram of water (a). Here, the three usual phases are
    distinguished, each separated from the other by a critical line. The phase
    transition that happens when the system traverse a line is characterized by
    a discontinuous jump in the density, as shown in (b) for the liquid-gas
    transition at $P=1$ atm. For $P=217.7$ atm however the same transition is
    continuous, as shown in (c). In the vicinity of this phase transition, at
    $T\approx374.4^\circ$C, the two phases become indistinguishable, and
    display a number of peculiarities. Systems in such state are called
    critical systems. Reproduced from~\cite{Sole2011}.}
\label{fig:water}
\end{figure}


\section{Classification of Phase Transitions}
\label{sec:classification}

Phase transitions are usually classified into two categories: first and second
order. There's no consensus on the precise definition of each, but this does
not matter a lot, since they behave quite differently. An old, but still often
used definition was proposed by Ehrenfest~\cite{Jaeger1998}. But, before we get
into that, we have to find a quantitative way of describing the phase of a
system. This is done by defining an \textit{order parameter}, a quantity that
take very distinct values in different phases (usually normalized to be zero in
the high temperature phase). In the case of water, the order parameter is the
density or, equivalently, the volume. For ferromagnetic transitions we use the
spontaneous magnetization. Other examples of order parameter can be seen in
Table~\ref{tab:ptex}.

Following the example of water, where the temperature and pressure are the
control variables, the thermodynamic properties are determined by its Gibbs
free energy
\begin{equation}
    G=E-TS+pV.
\end{equation}
from which we can obtain the volume by taking its derivative
\begin{equation}
    V={\left(\frac{\partial G}{\partial p}\right)}_T.
\end{equation}
If we now make an experimental apparatus to try and observe the behavior of the
density of an amount of water during a phase transition, say the liquid-gas
transition, what we see is shown in Figure~\ref{fig:water}b. The density (and
therefore the volume) makes a discontinuous jump. Less evident, but no less
true, is that the entropy, another derivative of $G$,
\begin{equation}
    S=-{\left(\frac{\partial G}{\partial T}\right)}_p,
\end{equation}
is also discontinuous at the boundary of a phase transition. Because the first
derivatives of the free energy display a singularity, we say these phase
transition are of \textit{first order}. This sudden changes happens because the
equilibrium state of the system (the minimum of the free energy) shifts from
one part of the phase space to another~\cite{Callen1985} as illustrated in
Figure~\ref{fig:gibbs1}. When the system is exactly at the boundary of the
phase transition, the two phases have the same free energy, allowing them both
to coexist in different regions of the system. Because of this, the transition
lines are also called coexistence lines. On the point labeled triple point, the
free energy has \textit{three} equal minima, so all three phases (solid, liquid
and gas) exist on the same system. Because the triple point is uniquely
defined, it is the basis used to define the Kelvin temperature
scale~\cite{Fermi1956}.

\begin{figure}
\begin{center}
    \includegraphics[scale=0.4]{chapters/ch2-crit/figs/gibbs1}
\end{center}
\caption{Why first order transitions happen. The left graph shows three points
    on the phase space of water. The red one is in the liquid phase, the blue
    is in the gaseous state and the green right on the boundary. The right
    graph shows a heuristic construction of the Gibbs free energy in each of
    the three points (matched by color). The equilibrium state is the one of
    minimal free energy. Although $G$ changes continuously with the
    temperature, the global minimum shifts abruptly once the phase boundary is
    crossed, thus occurring a first order phase transition. When the system is
    exactly on the boundary, the minima are equivalent, and the two phases
    coexist.}
\label{fig:gibbs1}
\end{figure}


First order transitions are characterized by the presence of latent heat, an
amount of energy that the system releases or absorbs during a phase transition,
which is a constant of the substance, and is independent on the temperature,
pressure or other control parameters~\cite{Callen1985}. This means that when
boiling a volume of water, the temperature of the liquid phase does not rise
because the extra heat being added is being absorbed in form of latent heat.

Now, looking again at Figure~\ref{fig:water}a one might ask why the liquid-gas
transition line does not extend for all values of temperature. This is where
things start to get interesting. We've established that the free energy have
two equivalent minima when the system is over the coexistence line. Now, if we
walk along this line following the behavior of the free energy, what we observe
is that the ``dividing wall'' between  the two minima gets shorter and shorter,
similar to what is shown in Figure~\ref{fig:gibbs2}. At a critical temperature
of approximately $374^\circ$C and pressure of $217.7$atm, the two minima merge.
At this point it is not that the two phases coexist, they are
indistinguishable from one another.

If we look at the order parameter when the system pass through the critical
point, we see that it is no longer discontinuous, as it's shown in
Figure~\ref{fig:water}c. If we set the pressure to the same as the critical
point and heat an amount of liquid water, the volume will rise continuously
until it reaches the gaseous state. Not all quantities are well behaved though.
The ones that do display discontinuities on the critical point, are the second
derivatives of the free energy, like the specific heat
\begin{equation}
    c_p=T{\left(\frac{\partial S}{\partial T}\right)}_{p}=
    T{\left(\frac{\partial^{2}G}{\partial T^{2}}\right)}_{p},
\end{equation}
or the isothermal compressibility
\begin{equation}
    \kappa_T=-\frac{1}{V}{\left(\frac{\partial V}{\partial p}\right)}_{T}=
    -\frac{1}{V}{\left(\frac{\partial^{2}G}{\partial p^{2}}\right)}_{T},
\end{equation}
This why going through a critical point is know as a \textit{second order}
phase transition, and systems close to it are called \textit{critical systems}.

Second order transitions have a whole range of peculiar behavior associated
with it. The most notorious is the presence of large fluctuations. In
Chapter~\ref{ch:intr} we mentioned the work of Thomas
Andrews~\cite{Andrews1869}, who noticed that some substances become opaque at
the critical point. This happens because the huge fluctuations in the
spacial distribution of mass, which happens in all size scales, scatter most
of the visible light.

Although we focused a lot on the example of water, the ferromagnetic transition
described in the previous section is probably the most studied kind of second
order phase transition. In it the free energy is given by
\begin{equation}
    F = -SdT-Mdh
\end{equation}
where $M$ is the spontaneous magnetization (the order parameter in this case)
and $h$ is an external magnetic field. It is characterized by a diverging
magnetic susceptibility
\begin{equation}
    \chi_{M}=
    {\left(\frac{\partial M}{\partial h}\right)}_{T}=
    -{\left(\frac{\partial^{2}F}{\partial h^{2}}\right)}_{T}.
\end{equation}

\begin{figure}
\begin{center}
    \includegraphics[scale=0.4]{chapters/ch2-crit/figs/gibbs2}
\end{center}
\caption{If we follow the (heuristic) Gibbs free energy along the liquid-gas
    line, we observe that at some critical point $T_c$ the two minima (related
    to the liquid and gas phases) coalesce into one. At this point the two
    phases are indistinguishable, which characterizes a second order phase
    transition. Systems close to the critical point are called critical
    systems.}
\label{fig:gibbs2}
\end{figure}


\begin{figure}
\begin{center}
    \includegraphics[scale=0.16]{chapters/ch2-crit/figs/suscep}
\end{center}
\caption{Examples of thermodynamical quantities in the vicinity of a second
    order phase transitions. The left graph shows the diverging magnetic
    susceptibility on the ferromagnetic transition of
    dupeyredioxyl~\cite{Chiarelli1993}. The graph on the right shows the
    diverging specific heat of the lambda transition of superfluid
    helium-4~\cite{Keesom1935}.}
\label{fig:suscep}
\end{figure}

Not all phase transitions fall neatly into these two categories, however. The
Kosterlitz-Thouless transition of the XY-Model (related to some superconducting
transitions~\cite{Resnick1981}) could be considered of infinite order, because
the free energy is infinitely differentiable~\cite{Kosterlitz1973}, although it
shares some of similarities with second order transitions.

\begin{table}
\newcolumntype{L}[1]{>{\raggedright\let\newline\\\arraybackslash\hspace{0pt}}m{#1}}
\begin{centering}
\begin{tabular}{L{4cm}L{5cm}L{2cm}L{2.0cm}L{0mm}}
\bottomrule[0.1mm]
\toprule[0.1mm]
\textbf{Transition Type}   & \textbf{Order Parameter}                         & \textbf{Example}          & $\mathbf{T_{c}}$ \textbf{(K)}   &\\
\bottomrule[0.1mm]
Liquid-gas                 & Molar volume                                     & H$_{2}$O                  & 647.05~\cite{Leveltsengers1974} &\\[0.5cm]
Ferromagnetic              & Magnetic moment                                  & Fe                        & 1044.0~\cite{Kadanoff1967}      &\\[0.5cm]
Antiferromagnetic          & Sublattice magnetic moment                       & FeF$_{2}$                 & 78.26~\cite{Ahlers1974}         &\\[0.5cm]
$\lambda$-line in $^{4}$He & $^{4}$He quantum mechanical amplitude            & $^{4}$He                  & 1.7--2.1~\cite{Ahlers1973}      &\\[0.5cm]
Superdonductivity          & Electron pair amplitude                          & Pb                        & 7.19~\cite{Heller1967}          &\\[0.5cm]
Binary fluid mixture       & Fractional segregation of components             & CCl$_{4}$-C$_{7}$F$_{14}$ & 301.78~\cite{Heller1967}        &\\[0.5cm]
Binary alloy               & Fraction of one atomic species on one sublattice & Cu-Zn                     & 739~\cite{Dietrich1966}         &\\[0.5cm]
Ferroelectric              & Electric dipole moment                           & Triglycine sulfate        & 322.5~\cite{Gonzalo1966}        &\\[0.5cm]
\bottomrule[0.1mm]
\toprule[0.1mm]
\end{tabular}
\par\end{centering}
\caption{Examples of systems that display a second order phase transition.
    Reproduced from~\cite{Shang2000}.}
\label{tab:ptex}
\end{table}



\section{Critical Exponents and Universality}
\label{sec:universality}

We noted that several thermodynamical quantities diverge on the critical point.
Not only that, that

We will use $\Phi$ for the order parameter (the density for water, the
magnetization for ferromagnets) and $J$ for the source field (pressure for
water, the external magnetic field for ferromagnets).

The heat capacity exponent $\alpha$
\begin{equation}
    \begin{array}{ccccc}
        C & \sim & {\left(T-T_c\right)}^{-\alpha}  & \mbox{for} & T > T_c \\
        C & \sim & {\left(T_c-T\right)}^{-\alpha'} & \mbox{for} & T < T_c
    \end{array}
\end{equation}
The order parameter exponent $\beta$, when taken along the coexistence curve 
\begin{equation}
    \begin{array}{ccccc}
        \Phi  & \sim & {\left(T_c-T\right)}^{\beta} & \mbox{for} & T < T_c
    \end{array}
\end{equation}
The generalized susceptibility exponent $\gamma$
\begin{equation}
    \begin{array}{ccccc}
        \chi  & \sim & {\left(T-T_c\right)}^{-\gamma}  & \mbox{for} & T > T_c \\
        \chi  & \sim & {\left(T_c-T\right)}^{-\gamma'} & \mbox{for} & T < T_c
    \end{array}
\end{equation}
The order parameter exponent $\delta$, when taken at constant temperature
$T=T_c$
\begin{equation}
    \begin{array}{ccccc}
        \Phi  & \sim & {\left(J-J_c\right)}^{1/\delta} & \mbox{for} & J > J_c
    \end{array}
\end{equation}
These exponents may have different values if you take the limit from above or
bellow the critical point, but it should be enough for us to assume they are
the same.

There are many other critical exponents outside the realm of thermodynamics and
well into the statistical mechanics domain. The most notorious one is related
to the correlations of the system. According to the fluctuation-dissipation
theorem
\begin{equation}
    \label{eq:flucdiss}
    \chi\sim
    \int\left\langle \Phi\left(0\right)\Phi\left(r\right)\right\rangle dr,
\end{equation}
where the term $\left\langle \Phi\left(0\right)\Phi\left(r\right)\right\rangle$
is called a correlation function. Here we are assuming translation invariance,
so $\left\langle \Phi\left(r_1\right)\Phi\left(r_2\right)\right\rangle$ depends
only on the distance $|r_1-r_2|$.

Because most systems studied do not have long range interactions, this means
the state of the components are determined by the ones closest to it. This means
that 
\begin{equation}
    \left\langle \Phi\left(0\right)\Phi\left(r\right)\right\rangle \sim e^{-r/\xi}
\end{equation}
where $\xi$ is called the \textit{correlation length}. As the temperature
rises, one would expect that the correlation length would get shorter, for the
thermal fluctuations but since the integral~\ref{eq:flucdiss} must diverge, the
correlation length must not stay finite. In fact it diverges with an exponent
$\nu$
\begin{equation}
    \xi\sim{\left|T-T_c\right|}^{-\nu}.
\end{equation}
and exactly above the critical point 
\begin{equation}
    \left\langle \Phi\left(0\right)\Phi\left(r\right)\right\rangle
    \sim r^{-d+2-\eta}
\end{equation}
where $d$ is the number of dimensions of the model. In order for the susceptibility
diverge, the condition $d+\eta\leq5$

The discussion of phase transitions presented (very superficially) so far,
rooted on the postulates of thermodynamics and in the analyticity of the free
energy, is due to Landau~\cite{Landau1969}. While this theory menages to
capture a few properties of critical systems, it fails catastrophically when it
comes to determine the values of the critical exponents. The source of this
discrepancy is due to some basic assumptions of statistical mechanics. You see,
the maximum entropy postulate is derived from the fact that the probability of
finding a system in any given state is distributed very narrowly around a
value, this way we can assume the expected (average) energy of the system is
the same as the most probable one. On the critical point however, because
the minimum of the free energy is so shallow, the probability distribution
significantly broadens, so much that the difference between most probable and
average cannot be ignored~\cite{Callen1985}.

One notable consequence of the dominance of long range correlations is that the
value of the critical exponents does not depend on the details of the material,
only by the much more general properties of the fluctuations. This means that
systems apparently unrelated, like superfluid helium or superconducting mercury,
will often have the same exponents ($\alpha\approx-0.0127$ for example).

\begin{table}
\newcolumntype{L}[1]{>{\raggedright\let\newline\\\arraybackslash\hspace{0pt}}m{#1}}
\begin{centering}
\begin{tabular}{r>{\raggedright}m{2.6cm}L{1.3cm}L{1.2cm}L{1.2cm}L{1.1cm}L{1.1cm}L{1.0cm}}
\bottomrule[0.1mm]
\toprule[0.1mm]
\textbf{Class}     & \textbf{Examples}   & \boldmath$\alpha$ & \boldmath$\beta$ & \boldmath$\gamma$ & \boldmath$\delta$ & \boldmath$\nu$ & \boldmath$\eta$  \\
\bottomrule[0.1mm]
Mean Field         & ---                 & $0$               & $1/2$            & $1$               & $3$               & $1/2$          & $0$     \\[0.5cm]
2D Ising           & Ferromagnet         & $0$               & $1/8$            & $7/4$             & $15$              & $1$            & $1/4$   \\[0.5cm]
3D Ising           & Liquid-Vapor        & $0.11$            & $0.33$           & $1.24$            & $4.79$            & $0.63$         & $0.036$ \\[0.5cm]
3D XY-Model        & Superfluid $^{4}$He & $-0.015$          & $0.348$          & $1.31$            & $4.78$            & $0.671$        & $0.038$ \\[0.5cm]
2D 3-State Potts   & Cubic ferromagnet   & $1/3$             & $1/9$            & $13/9$            & $14$              & $5/6$          & $4/15$  \\[0.5cm]
2D Percolation     & Porous Media        & $-2/3$            & $5/36$           & $43/18$           & $91/5$            & $4/3$          & $5/24$  \\[0.5cm]
\bottomrule[0.1mm]
\toprule[0.1mm]
\end{tabular}
\par\end{centering}
\caption{Non exaustive list of critical exponents of various universlity
    classes~\cite{Callen1985, Onsager1944, ElShowk2014, Guillou1977, Wu1982,
    Smirnov2001}. Thke mean field is the result found using the Landau approach
    shown in Section~\ref{sec:classification}, it is an approximation of high
    dimensional systems.}
\label{tab:univ}
\end{table}


\section{Models of Critical Systems}
\label{sec:models}

Here we will show and discuss two of the most popular models of critical systems.

\subsection{Ising Model}
\label{sec:ising}

Although not directly relevant for the present work, we shall describe briefly
the Ising model if only because of its importance in the field of critical
phenomena.


\subsection{Percolation}
\label{sec:perc}

While the Ising model certainly wins in terms of popularity, very few models
match the simplicity of percolation. Introduced in 1957 by Broadbent and
Hammersley, the passing decades saw its rise in prominence due to it's high
applicability ranging from transport in porous media to the propagation of
infectious diseases, with pretty much everything in between~\cite{Araujo2013}.
