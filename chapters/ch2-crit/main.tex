\chapter{Phase Transitions and Critical Phenomena}
\label{ch:crit}

When trying to analyze a physical system, a big part of a physicist effort is
dedicated to identifying the parts that compose the system and how they
interact with each other. It is usually a daunting task to find an interaction
model that captures all the relevant properties of a system. That's because the
macroscopic (i.e.\ collective) behavior does not follow trivially from the
microscopic forces in place. In no area of physics this is more clear than in
that of phase transitions and critical phenomena. What's so special about the
critical point, that allows a set of simple building blocks, which contain no
global information about the system, display such a distinct and complex large
scale behavior? This is the question scientists have been trying to solve in
the last century or so, with each generation providing significant
contributions to this end.

%Not surprisingly the
%relationship between criticality and complexity has been applied to areas well
%beyond the domain of atoms and molecules, like cognitive
%sciences~\cite{Kello2010}, social scieces~\cite{Kron2009} and
%ecology~\cite{Sole1999}, to name just a few.

To start, let's take a step back and look at the constituent parts of a system.
We can get away with making very little assumptions about their actual
identities. They can be elementary particles when studying high energy physics,
atoms and molecules when determining the electronic properties of a material,
living cells when observing the growing pattern of a colony of bacteria, or
even people when trying to design buildings with safe exit routes. What is more
important to us is the fact that these building blocks can organize themselves
in several different ways. Each pattern of organization is called a
\textit{phase}, and in which phase the systems finds itself depends on the
value of some external parameters, like energy density, temperature, nutrient
concentration or whether or not there's a fire in the building. Not only that
but different phases of a same system display very different macroscopic
behavior, despite no change taking place in their actual composition.

The most elementary example here is water. At sea level, water is in its solid
phase for temperatures lower than $0^\circ$C. That happens because the water
molecules are locked in place by the cohesive forces that its neighbors exert
on each other, forming a crystalline structure. When the temperature is higher
than $0^\circ$C but lower than $100^\circ$C, the molecules acquire enough
kinetic energy to be able to move through one another. They still interact
strongly enough to be bound together but it is not sufficient to keep them
fixed. This is the liquid phase. When the temperature is above $100^\circ$C,
water is now in the gaseous state, where thermal movement dominates and the
molecules move mostly unimpeded, interacting only when colliding directly. See
Figure~\ref{fig:phases} for an illustration of how a molecule moves in the
three phases.

Not all phases concern the states of matter, however. Some are related to
magnetic or electric properties. This is the case of ferromagnetic materials,
materials that exhibit a persistent magnetization after being subject to a
strong magnetic field, like your everyday fridge magnet. This phenomenon is a
result of the alignment of the magnetic spins of the atoms or molecules that
compose the material. In contrast, paramagnetic materials show no permanent
magnetization. If you take a magnet and raise its temperature, eventually the
thermal fluctuations will misalign the spins, zeroing out the magnetization and
turning the ferromagnetic material into a paramagnetic one. Other examples of
phase transitions also include the transition of a material into a
superconductive state~\cite{Fisher1991} or when helium turns into into a
superfluid~\cite{Campostrini2006}, both happening at very low temperatures.

To better visualize the different phases of systems, it is common to make use
of a \textit{phase diagram}. Simply put, a phase diagram is a graph where each
axis represent one control parameter, and every point in this graph is labeled
according to the phase the system finds itself for that value of the
parameters. Figure~\ref{fig:water}a shows the very famous phase diagram of
water as a function of temperature and pressure. The three main phases (solid,
liquid and gas) are clearly marked, and the black lines are the boundary
between the phases. Crossing any of these lines is followed by a sharp change
on the properties of the system, as described previously. This is know as a
\textit{phase transition}.

\begin{figure}
\begin{center}
    \includegraphics[width=\textwidth]{chapters/ch2-crit/figs/phases}
\end{center}
\caption{How a system of particles behave in the three phases of matter. Here a
    simulation of 100 particles was done, but the trajectory of only one is
    shown. In the solid state the particles are confined to small region by
    the interactions of its neighbors. In the liquid state the particle is
    unconfined, but still interacts strongly with the other particles. In the
    gas state the kinetic energy of the particles is large enough that they
    barely interact with one another, making a ballistic trajectory until
    they make a head-on collision with another particle.}
\label{fig:phases}
\end{figure}


\begin{figure}
\begin{center}
    \includegraphics[width=\textwidth]{chapters/ch2-crit/figs/water}
\end{center}
\caption{Phase diagram of water (a). Here, the three usual phases are
    distinguished, each separated from the other by a coexistence line. The
    phase transition that happens when the system traverse a line is
    characterized by a discontinuous jump in the density, as shown in (b) for
    the liquid-gas transition at $P=1$ atm. For $P=217.7$ atm however the same
    transition is continuous, as shown in (c). In the vicinity of this phase
    transition, at $T\approx374.4^\circ$C, the two phases become
    indistinguishable, and display a number of peculiarities. Systems in such
    state are called critical systems. Reproduced from~\cite{Sole2011}.}
\label{fig:water}
\end{figure}


\section{Classification of Phase Transitions}
\label{sec:classification}

Phase transitions are usually classified into two categories: first and second
order. There's no consensus on the precise definition of each, but this does
not matter a lot, since they behave quite differently. An old, but still often
used definition was proposed by Ehrenfest~\cite{Jaeger1998}. But, before we get
into that, we have to find a quantitative way of describing the phase of a
system. This is done by defining an \textit{order parameter}, a quantity that
take very distinct values in different phases (usually normalized to be zero in
the most disordered phase). In the case of water, the order parameter is the
density (which is just the inverse of the volume). For ferromagnetic
transitions we use the spontaneous magnetization. The choice of order parameter
for several systems can be seen in Table~\ref{tab:ptex}.

Following the example of water, where the temperature and pressure are the
control variables, the thermodynamic properties are determined by the Gibbs
free energy
\begin{equation}
    G=E-TS+pV.
\end{equation}
from which we can obtain the volume by taking its derivative
\begin{equation}
    V={\left(\frac{\partial G}{\partial p}\right)}_T.
\end{equation}
If we now make an experimental apparatus to try and observe the behavior of the
density of an amount of water during a phase transition, say the liquid-gas
transition, what we see is shown in Figure~\ref{fig:water}b. The density (and
therefore the volume) makes a discontinuous jump. Less evident, but no less
true, is that the entropy, another derivative of $G$,
\begin{equation}
    S=-{\left(\frac{\partial G}{\partial T}\right)}_p,
\end{equation}
is also discontinuous at the boundary of a phase transition. Because the first
derivatives of the free energy display a singularity, we say these phase
transition are of \textit{first order}. This sudden changes happens because the
equilibrium state of the system (determined by the minimum of the free energy)
shifts from one part of the phase space to another~\cite{Callen1985}, as
illustrated in Figure~\ref{fig:gibbs1}. When the system is exactly at the
boundary of the phase transition, the two phases have the same free energy,
allowing them both to coexist in different regions of the system. Because of
this, the transition lines are also called coexistence lines. On the point
labeled triple point, the free energy has \textit{three} equal minima, so all
three phases (solid, liquid and gas) exist on the same system. Because the
triple point is uniquely defined, it is the basis used to define the Kelvin
temperature scale~\cite{Fermi1956}.

\begin{figure}
\begin{center}
    \includegraphics[width=\textwidth]{chapters/ch2-crit/figs/gibbs1}
\end{center}
\caption{Why first order transitions happen. The left graph shows three points
    on the phase space of water. The red one is in the liquid phase, the blue
    is in the gaseous state and the green right on the boundary. The right
    graph shows a heuristic construction of the Gibbs free energy in each of
    the three points (matched by color). The equilibrium state is the one of
    minimal free energy. Although $G$ changes continuously with the
    temperature, the global minimum shifts abruptly once the phase boundary is
    crossed, thus occurring a first order phase transition. When the system is
    exactly on the boundary, the minima are equivalent, and the two phases
    coexist.}
\label{fig:gibbs1}
\end{figure}


First order transitions are characterized by the presence of latent heat, an
amount of energy that the system releases or absorbs during a phase transition,
which is a constant of the substance, and is independent on the temperature,
pressure or other control parameters~\cite{Callen1985}. This means that when
boiling a volume of water, the temperature of the liquid phase does not rise
because the extra heat being added is being absorbed in form of latent heat.

Now, looking again at Figure~\ref{fig:water}a one might ask why the liquid-gas
transition line does not extend all the way through the phase diagram. This
is where things start to get interesting. We've established that the free
energy have two equivalent minima when the system is over the coexistence line.
Now, if we walk along this line following the behavior of the free energy, what
we observe is that the ``dividing barrier'' between  the two minima gets shorter
and shorter, similar to what is shown in Figure~\ref{fig:gibbs2}. At a critical
temperature of approximately $374^\circ$C and pressure of $217.7$atm, the two
minima merge. At this point it is not that the two phases coexist, they are
indistinguishable from one another.

If we look at the order parameter when the system pass through the critical
point, we see that it is no longer discontinuous, as it's shown in
Figure~\ref{fig:water}c. If we set the pressure to the same as the critical
point and heat an amount of liquid water, the volume will rise continuously
(although not necessarily smoothly) until it reaches the gaseous state. Not all
quantities are well behaved though. The ones that do display discontinuities on
the critical point, are the second derivatives of the free energy, like the
specific heat
\begin{equation}
    c_p=T{\left(\frac{\partial S}{\partial T}\right)}_{p}=
    T{\left(\frac{\partial^{2}G}{\partial T^{2}}\right)}_{p},
\end{equation}
or the isothermal compressibility
\begin{equation}
    \kappa_T=-\frac{1}{V}{\left(\frac{\partial V}{\partial p}\right)}_{T}=
    -\frac{1}{V}{\left(\frac{\partial^{2}G}{\partial p^{2}}\right)}_{T},
\end{equation}
This why going through a critical point is know as a \textit{second order}
phase transition, and systems close to it are called \textit{critical systems}.

Second order transitions have a whole range of peculiar behavior associated
with it. The most notorious is the presence of large fluctuations. In
Chapter~\ref{ch:intr} we mentioned the work of Thomas
Andrews~\cite{Andrews1869}, who noticed that some substances become opaque at
the critical point. This happens because the huge fluctuations in the spacial
distribution of mass, forming ``blobs'' of water of all sizes, including the
sizes that scatter most visible light, giving the system a milky aspect.

Although we focused a lot on the example of water, the ferromagnetic transition
described in the previous section is probably the most studied kind of second
order phase transition. In it the free energy is given by
\begin{equation}
    F = -SdT-Mdh
\end{equation}
where $M$ is the spontaneous magnetization (the order parameter in this case)
and $h$ is an uniform external magnetic field. It is characterized by a
diverging magnetic susceptibility
\begin{equation}
    \chi_{M}=
    {\left(\frac{\partial M}{\partial h}\right)}_{T}=
    -{\left(\frac{\partial^{2}F}{\partial h^{2}}\right)}_{T}.
\end{equation}

\begin{figure}
\begin{center}
    \includegraphics[width=\textwidth]{chapters/ch2-crit/figs/gibbs2}
\end{center}
\caption{If we follow the (heuristic) Gibbs free energy along the liquid-gas
    line, we observe that at some critical point $T_c$ the two minima (related
    to the liquid and gas phases) coalesce into one. At this point the two
    phases are indistinguishable, which characterizes a second order phase
    transition. Systems close to the critical point are called critical
    systems.}
\label{fig:gibbs2}
\end{figure}


\begin{figure}
\begin{center}
    \includegraphics[width=\textwidth]{chapters/ch2-crit/figs/suscep}
\end{center}
\caption{Examples of thermodynamical quantities in the vicinity of a second
    order phase transitions. The left graph shows the diverging magnetic
    susceptibility on the ferromagnetic transition of
    dupeyredioxyl~\cite{Chiarelli1993}. The graph on the right shows the
    diverging specific heat of the lambda transition of superfluid
    helium-4~\cite{Keesom1935}.}
\label{fig:suscep}
\end{figure}

Not all phase transitions fall neatly into these two categories, however. The
Kosterlitz-Thouless transition of the XY-Model (related to some superconducting
transitions~\cite{Resnick1981}) could be considered of infinite order, because
the free energy is infinitely differentiable~\cite{Kosterlitz1973}, although it
shares some of similarities with second order transitions.

The discussion of phase transitions presented (very superficially) so far,
rooted on the postulates of thermodynamics and in the analyticity of the free
energy, is due to Landau~\cite{Landau1969}. While this theory manages to
capture a few properties of critical systems, it fails catastrophically from a
quantitative standpoint. The source of this discrepancy comes from some basic
assumptions of statistical mechanics. You see, the maximum entropy postulate is
derived from the fact that the probability of finding a system in any given
state is distributed very narrowly around a value, this way we can assume the
expected (average) energy of the system is the same as the most probable one.
On the critical point however, because the minimum of the free energy is so
shallow, the probability distribution significantly broadens, so much that the
difference between most probable and average cannot be
ignored~\cite{Callen1985}.

\begin{table}
\newcolumntype{L}[1]{>{\raggedright\let\newline\\\arraybackslash\hspace{0pt}}m{#1}}
\begin{centering}
\begin{tabular}{L{4cm}L{5cm}L{2cm}L{2.0cm}L{0mm}}
\bottomrule[0.1mm]
\toprule[0.1mm]
\textbf{Transition Type}   & \textbf{Order Parameter}                         & \textbf{Example}          & $\mathbf{T_{c}}$ \textbf{(K)}   &\\
\bottomrule[0.1mm]
Liquid-gas                 & Molar volume                                     & H$_{2}$O                  & 647.05~\cite{Leveltsengers1974} &\\[0.5cm]
Ferromagnetic              & Magnetic moment                                  & Fe                        & 1044.0~\cite{Kadanoff1967}      &\\[0.5cm]
Antiferromagnetic          & Sublattice magnetic moment                       & FeF$_{2}$                 & 78.26~\cite{Ahlers1974}         &\\[0.5cm]
$\lambda$-line in $^{4}$He & $^{4}$He quantum mechanical amplitude            & $^{4}$He                  & 1.7--2.1~\cite{Ahlers1973}      &\\[0.5cm]
Superdonductivity          & Electron pair amplitude                          & Pb                        & 7.19~\cite{Heller1967}          &\\[0.5cm]
Binary fluid mixture       & Fractional segregation of components             & CCl$_{4}$-C$_{7}$F$_{14}$ & 301.78~\cite{Heller1967}        &\\[0.5cm]
Binary alloy               & Fraction of one atomic species on one sublattice & Cu-Zn                     & 739~\cite{Dietrich1966}         &\\[0.5cm]
Ferroelectric              & Electric dipole moment                           & Triglycine sulfate        & 322.5~\cite{Gonzalo1966}        &\\[0.5cm]
\bottomrule[0.1mm]
\toprule[0.1mm]
\end{tabular}
\par\end{centering}
\caption{Examples of systems that display a second order phase transition.
    Reproduced from~\cite{Shang2000}.}
\label{tab:ptex}
\end{table}


\section{Critical Exponents and Universality}
\label{sec:universality}
\newcommand{\op}{\sigma}
\newcommand{\sfi}{J}

We noted that several thermodynamical quantities diverge on the critical point.
More important than that is to know \textit{how} they diverge. This motivates
the definition of a set of indices called \textit{critical exponents} that
characterize the behavior of the system near the critical point. Here we will
use $\op$ for the order parameter (the density for water, the magnetization for
ferromagnets) and $\sfi$ for the source field (pressure for water, the external
magnetic field for ferromagnets). This way we can define a generalized
susceptibility $\chi=\partial\op/\partial \sfi$.

In the thermodynamics domain there are four main critical exponents, each
associated with a different quantity. We have the heat capacity exponent
$\alpha$
\begin{equation}
    \label{eq:heatc}
    \begin{array}{ccccc}
        c & \sim & {\left(T-T_c\right)}^{-\alpha}  & \mbox{for} & T > T_c \\
        c & \sim & {\left(T_c-T\right)}^{-\alpha'} & \mbox{for} & T < T_c,
    \end{array}
\end{equation}
the order parameter exponent $\beta$, when taken along the coexistence curve 
\begin{equation}
    \begin{array}{ccccc}
        \op  & \sim & {\left(T_c-T\right)}^{\beta} & \mbox{for} & T < T_c,
    \end{array}
\end{equation}
the generalized susceptibility exponent $\gamma$
\begin{equation}
    \begin{array}{ccccc}
        \chi  & \sim & {\left(T-T_c\right)}^{-\gamma}  & \mbox{for} & T > T_c \\
        \chi  & \sim & {\left(T_c-T\right)}^{-\gamma'} & \mbox{for} & T < T_c,
    \end{array}
\end{equation}
and the order parameter exponent $\delta$, when taken at constant temperature
$T=T_c$
\begin{equation}
    \begin{array}{ccccc}
        \op  & \sim & {\left(\sfi-\sfi_c\right)}^{1/\delta} & \mbox{for} & \sfi > \sfi_c.
    \end{array}
\end{equation}
These exponents may have different values if you take the limit from above or
bellow the critical point (unprimed and primed symbols respectively), but
unless explicitly noted, we'll assume that they're the same, which happens
often.

There are many other critical exponents outside the realm of thermodynamics.
These exponents can only be studied by a full statistical mechanical framework.
Two of the most important ones are related to the correlations of the system.

Take the ferromagnetic transition, for instance. The influence of an individual
spin is usually restricted over a finite region around it. That is to say,
there is a length scale $\xi$ that if we take two spins separated by a distance
$r\ll\xi$ and flip one of the spins, the other is very likely to also flip. In
the unordered (paramagnetic) phase, $\xi$ is small because the thermal
fluctuations inhibit long range correlations. On the ordered (ferromagnetic)
phase, the influence of large clusters of aligned spins is stronger than that
of a single one, also hindering the range of correlations. We see that too
much order, as well as too much disorder, have the effect of restricting the
influence of the spins over each another. Therefore, it makes sense to think
that somewhere in between there is a sweet spot where the balance between order
and disorder makes the correlation length as large as possible.

We can formalize this thought by defining the correlation function of a system,
\begin{equation}
    \label{eq:corr}
    C\left(x_{1},x_{2}\right)=
    C\left(\left|x_{1}-x_{2}\right|\right)=
    \left\langle \op\left(x_{1}\right)\op\left(x_{2}\right)\right\rangle 
    -\left\langle \op\left(x_{1}\right)\right\rangle
    \left\langle \op\left(x_{2}\right)\right\rangle,
\end{equation}
where $\op(x)$ is the local value of the order parameter at any given point of
the system. The macroscopic value can be recovered from its expected value
$\op=V^{-1}\int\op(x)d^d x$. The assumption that
$C\left(x_{1},x_{2}\right)=C\left(\left|x_{1}-x_{2}\right|\right)$ comes from
translation invariance, which takes no specific point of the system as special.
It is natural to expect that the correlations would decay exponentially, with a
decay constant $\xi$
\begin{equation}
    \label{eq:corr}
    C\left(r\right)\sim e^{-r/\xi}.
\end{equation}

The relation between thermodynamics and the correlation function is governed
by the fluctuation-dissipation theorem
\begin{equation}
    \chi=\frac{\partial\op}{\partial \sfi}=\frac{1}{T}\int C\left(r\right)dr.
\end{equation}
We know that on the critical point $\chi$ diverges, but  the integral on the
right-hand side cannot diverge for finite $\xi$, given Eq.~\ref{eq:corr}.
The conclusion is simple, not only the correlation length is maximal at the
critical point, it is infinite! This divergence also has a critical exponent
$\nu$ associated with it,
\begin{equation}
    \label{eq:corlen}
    \begin{array}{ccccc}
        \xi & \sim & {\left(T-T_c\right)}^{-\nu}  & \mbox{for} & T > T_c \\
        \xi & \sim & {\left(T_c-T\right)}^{-\nu'} & \mbox{for} & T < T_c.
    \end{array}
\end{equation}
The correlation function itself also cannot decay as an exponential, and goes
instead as a power law, with yet another critical exponent $\eta$,
\begin{equation}
    \label{eq:critcor}
    \begin{array}{ccccc}
        C(r) & \sim & r^{-d+2-\eta} & \mbox{for} & T = T_c.
    \end{array}
\end{equation}
where $d$ is the number of spatial dimensions (usually 2 or 3, but it is often
useful to leave it as a variable parameter~\cite{Wilson1972}). This is no minor
fact, it means that near the critical point, all length scales contribute to
the overall behavior of the system, making it that much more complicated to
treat. Unlike, say, in fluid dynamics, where you don't need to know the details
about the intermolecular interaction of water molecules to predict the ocean
tides, in critical systems the large and small length scales are indissociable.
Many authors consider this to be the defining characteristic of a second order
phase transition.

The most notable consequence of the dominance of long range correlations is
that the critical behavior of the system is determined by the general
properties of the fluctuations, which in turn is mainly determined by the
dimensionality of the system and of the order parameter, and not by the details
of the material that compose the system~\cite{Stanley1999}. As a consequence,
all systems tend to group themselves into classes that share the same critical
exponents. These are called \textit{universality classes}. For example, the
liquid-gas transition falls in the same universality class as the ferromagnetic
transition~\cite{Kim1984}. Table~\ref{tab:univ} show some examples of
universality classes, systems that belong to it, and the values of their
critical exponents.

\begin{table}
\newcolumntype{L}[1]{>{\raggedright\let\newline\\\arraybackslash\hspace{0pt}}m{#1}}
\begin{centering}
\begin{tabular}{r>{\raggedright}m{2.6cm}L{1.3cm}L{1.2cm}L{1.2cm}L{1.1cm}L{1.1cm}L{1.0cm}}
\bottomrule[0.1mm]
\toprule[0.1mm]
\textbf{Class}     & \textbf{Examples}   & \boldmath$\alpha$ & \boldmath$\beta$ & \boldmath$\gamma$ & \boldmath$\delta$ & \boldmath$\nu$ & \boldmath$\eta$  \\
\bottomrule[0.1mm]
Mean Field         & ---                 & $0$               & $1/2$            & $1$               & $3$               & $1/2$          & $0$     \\[0.5cm]
2D Ising           & Ferromagnet         & $0$               & $1/8$            & $7/4$             & $15$              & $1$            & $1/4$   \\[0.5cm]
3D Ising           & Liquid-Vapor        & $0.11$            & $0.33$           & $1.24$            & $4.79$            & $0.63$         & $0.036$ \\[0.5cm]
3D XY-Model        & Superfluid $^{4}$He & $-0.015$          & $0.348$          & $1.31$            & $4.78$            & $0.671$        & $0.038$ \\[0.5cm]
2D 3-State Potts   & Cubic ferromagnet   & $1/3$             & $1/9$            & $13/9$            & $14$              & $5/6$          & $4/15$  \\[0.5cm]
2D Percolation     & Porous Media        & $-2/3$            & $5/36$           & $43/18$           & $91/5$            & $4/3$          & $5/24$  \\[0.5cm]
\bottomrule[0.1mm]
\toprule[0.1mm]
\end{tabular}
\par\end{centering}
\caption{Non exaustive list of critical exponents of various universlity
    classes~\cite{Callen1985, Onsager1944, ElShowk2014, Guillou1977, Wu1982,
    Smirnov2001}. Thke mean field is the result found using the Landau approach
    shown in Section~\ref{sec:classification}, it is an approximation of high
    dimensional systems.}
\label{tab:univ}
\end{table}



\section{Scaling Invariance}
\label{sec:scaling}
\renewcommand{\op}{\phi}

The list of critical exponents shown on the last section is far from
exhaustive. More exotic, but no less important exponents include the fractal
dimension, one-arm exponent and the crossing exponent, among others. It sure
seems like a lot of work to completely describe a universality class, if one
needs to compute dozens of exponents. Luckily with the very simple assumption
of scaling invariance, we can greatly reduce the amount of work needed.

We will take a look at the correlation functions of the observables, which
we'll refer as \textit{scaling operators}. The main ones are the order
parameter density $\op(\mathbf{r})$ and energy density
$\varepsilon(\mathbf{r})$. The two-point correlation function of these fields is
defined according to Eq.~\ref{eq:corr}
\begin{eqnarray}
    \label{eq:corr10}
    C_{\op}\left(\left|\mathbf{r_1}-\mathbf{r_2}\right|\right) & = &
    \left\langle
        \op\left(\mathbf{r_{1}}\right)
        \op\left(\mathbf{r_{2}}\right)
    \right\rangle -
    \left\langle
        \op\left(\mathbf{r_{1}}\right)
    \right\rangle
    \left\langle
        \op\left(\mathbf{r_{2}}\right)
    \right\rangle
    \\
    \label{eq:corr11}
    C_{\varepsilon}\left(\left|\mathbf{r_1}-\mathbf{r_2}\right|\right) & = &
    \left\langle
        \varepsilon\left(\mathbf{r_{1}}\right)
        \varepsilon\left(\mathbf{r_{2}}\right)
    \right\rangle-
    \left\langle
        \varepsilon\left(\mathbf{r_{1}}\right)
    \right\rangle
    \left\langle
        \varepsilon\left(\mathbf{r_{2}}\right)
    \right\rangle.
\end{eqnarray}
They are given in terms of the reduced temperature and source field
\begin{equation}
    t=\frac{T-T_c}{T_c},\;\;\;\;\;\;\;\;\;h=\frac{\sfi-\sfi_c}{\sfi_c},
\end{equation}
so the critical point is exactly at $t=0$ and $h=0$. These are also known
as \textit{scaling fields}.

The scaling hypothesis postulates that the correlation functions transform
under a scaling transformation $\mathbf{r}\rightarrow\mathbf{r}/b$ in the
following way
\begin{eqnarray}
    \label{eq:scal}
    C_{\op}\left(\mathbf{r};t,h\right) & = &
    b^{-2x_{\op}}C_{\op}\left(\mathbf{r}/b;tb^{y_{t}},hb^{y_{h}}\right)\\
    \label{eq:scal10}
    C_{\varepsilon}\left(\mathbf{r};t,h\right) & = &
    b^{-2x_{\varepsilon}}C_{\varepsilon}\left(\mathbf{r}/b;tb^{y_{t}},hb^{y_{h}}\right)
\end{eqnarray}
where $x_\op$ and $x_\varepsilon$ are the scaling dimension of the scaling
operators $\op$ and $\varepsilon$, and $y_t$ and $y_h$ are called the
\textit{renormalization group eigenvalues} of their respective scaling fields.

Following the fluctuation dissipation theorem, we can compute the susceptibility
and the specific heat by integrating $C_\op$ and $C_\varepsilon$ respectively.
This give us
\begin{eqnarray}
    \label{eq:chi2}
    \chi\left(t,h\right) & = & b^{d-2x_{\op}}\chi\left(tb^{y_{t}},ht^{y_{h}}\right)\\
    \label{eq:c2}
    c\left(t,h\right) & = & b^{d-2x_{\varepsilon}}c\left(tb^{y_{t}},ht^{y_{h}}\right)
\end{eqnarray}
Recall that $\chi=-\partial^2 f/\partial h^2$ and $c=-\partial^2 f/\partial
T^2$, therefore we can integrate Eq.~\ref{eq:chi2} and~\ref{eq:c2} twice,
obtaining the free energy density $f=F/\mathcal{N}$
\begin{equation}
    f\left(t,h\right)=
    b^{d-2x_{\op}-2y_{h}}f\left(tb^{y_{t}},ht^{y_{h}}\right)=
    b^{d-2x_{\varepsilon}-2y_{t}}f\left(tb^{y_{t}},ht^{y_{h}}\right),
\end{equation}
from which we can deduce that $x_\op+y_h = x_\varepsilon+y_t$. As a matter of
fact, the sum is constant for all pairs of scaling dimension and its conjugate
renormalization group eigenvalue. Since this sum does not depend on the fields
themselves, it is safe to assume they depend on the only other parameter, the
number of spatial dimensions, that is
\begin{equation}
    \label{eq:scald}
    x_\op+y_h = x_\varepsilon+y_t = d
\end{equation}
which leaves us with the neat relation
\begin{equation}
    f\left(t,h\right)=b^{-d}f\left(tb^{y_{t}},ht^{y_{h}}\right).
\end{equation}
It is worth noting that this is not really the free energy of the system. The
actual free energy have a singular and a regular part, the one described so
far is the singular part. Near the critical point, however, only the singular
part makes significant contributions to the behavior of the system, so we can
safely ignore the regular part.

The next step is to make the scaling parameter $b$ a function of the
temperature, in order to get rid of one of the arguments of the free energy.
This can be done by making $b=t^{-1/y_t}$, which yields
\begin{equation}
    f\left(t,h\right)=
    t^{d/y_{t}}f\left(1,ht^{-y_{h}/y_{t}}\right)=
    t^{d/y_{t}}\Psi\left(ht^{-y_{h}/y_{t}}\right),
\end{equation}
where $\Psi$ is known as a scaling function, and it unique for a given
universality class. We can re-obtain the heat capacity by taking
the second derivative as usual
\begin{equation}
    c=-\left.\frac{\partial^2 f}{\partial t^2}\right|_{h=0}
    \sim \left|t\right|^{d/y_h-2}.
\end{equation}
Which is the expected behavior, and from Eq.~\ref{eq:heatc} we have that
$\alpha=2-d/y_t$. A similar process can be done to derive a similar relation
for all four thermodynamical exponents
\begin{eqnarray}
    \label{eq:scal1}
    \alpha & = & 2-d/y_{t}\\
    \beta  & = & \left(d-y_{h}\right)/y_{t}\\
    \gamma & = & \left(2y_{h}-d\right)/y_{t}\\
    \label{eq:scal2}
    \delta & = & y_{h}/\left(d-y_{h}\right).
\end{eqnarray}
This system of equations can be solved to single out the eigenvalues obtaining
the relations
\begin{eqnarray}
    \alpha+2\beta+\gamma              & = & 2\\
    \alpha+\beta\left(1+\delta\right) & = & 2,
\end{eqnarray}
These are known as the Rushbrooke equality~\cite{Rushbrooke1963} and Griffiths
equality~\cite{Griffiths1967} respectively,  which are some of the greatest
results of the scaling hypothesis, if you know the value of two of the
thermodynamical exponents, you get the other two for free.

How about the other two exponents, $\nu$ and $\eta$, related to the
correlations of the order parameter, you might ask. We can also find a simple
relation for them by taking Eq.~\ref{eq:scal} and again making $b=t^{-1/y_t}$
\begin{equation}
    C_{\op}\left(\mathbf{r};t,0\right)=
    t^{2x_{\op}/y_{t}}C_{\op}
    \left(\frac{\mathbf{r}}{{\left(1/t\right)}^{1/y_{t}}};1,0\right).
\end{equation}
We can assume that close to criticality (but not exactly on the critical point)
the correlation function depends only on $r/\xi$, because of
Eq.~\ref{eq:corr}. From this follows that
\begin{equation}
    \xi\sim\left|t\right|^{-1/y_t},
\end{equation}
which, using Eq.~\ref{eq:corlen}, implies that
\begin{equation}
    \label{eq:nuscal}
    \nu=\frac{1}{y_t}.
\end{equation}
Now we can repeat the process by making $b=r$ and $t=0$, which will give us
\begin{equation}
    C_{\op}\left(r;0,0\right)=r^{-2d+2y_{h}}.
\end{equation}
Comparing with Eqs.~\ref{eq:critcor} and~\ref{eq:scald}, gives the relation for
$\eta$
\begin{equation}
    \label{eq:etascal}
    \eta=d-2y_h+2.
\end{equation}
Finally we can find a relation between $\nu$ and $\eta$ using
Eqs.~\ref{eq:scal1}--\ref{eq:scal2},~\ref{eq:nuscal} and~\ref{eq:etascal}.
\begin{eqnarray}
    \label{eq:hs1}
    \gamma & = & \nu\left(2-\eta\right)\\
    \label{eq:hs2}
    \alpha & = & 2-d\nu.
\end{eqnarray}
These are known as \textit{hyperscaling relations}, and amazingly no new
critical exponent is needed to determine them, that is, only two are necessary
to completely describe the critical exponents of a second-order phase
transition, how convenient. Sadly the hyperscaling relations are not as sturdy
as the scaling relations and can break down under the presence of (the
creatively named) dangerously irrelevant operators~\cite{Nishimori2011}.

The arguments and assumptions used throughout this section were historically
motivated through the study of renormalization group
theory~\cite{Pelissetto2002}, which was an important advancement in critical
phenomena when it was first developed, but it goes well beyond the scope of
this work. However, some authors do prefer to look at scaling invariance
through an axiomatic lens, as it was done here, forming a parallel with the
introduction of conformal invariance (see
Section~\ref{ch:conf})~\cite{Henkel2013}.


\section{Models of Critical Systems}
\label{sec:models}

Beyond the phenomenological approach shown up to this point, another way to
advance the knowledge of statistical physics is through model building, where
we look for simple models capable of providing useful insight into practical
situations. Coming up with a fruitful model however can be a very tiresome and
sometimes arcane task, with a lot of trial and error involved. Over decades of
study, a select group of models have gained a significant status in the canon
of phase transitions. Two that stand out due to their simplicity and historical
significance are the Ising and percolation models.


\subsection{Ising Model}
\label{sec:ising}

Of all phase transition models, Ising's is definitely the most famous of all.
Its importance come not only from its simplicity, but from the fact that it's
one of the 

Today the Ising model forms the basis of our understanding of phase transitions
to the point that many latter important models, like the $O(n)$ and Potts
models~\cite{Wu1982}, can be traced back to it.

Although not directly relevant for the present work, we shall briefly describe
the Ising model because of its importance in the field of critical phenomena.
It's hard to understate the its impact in the study of phase transitions. It
represents a golden standard because it unites an elegantly simple setup with
the fact it can be exactly solved in both one and two dimensions, a claim very
few models can make. Some of the most important models,

The system is composed of a number of classical spins $\{s_i\}$ that can take
one of two values $1$ or $-1$. They are arranged in a lattice and are allowed
to interact with its first nearest neighbors. The Hamiltonian of the systems is
given by
\begin{equation}
    \mathcal{H}=
    -J\sum_{\left\langle i,j\right\rangle }s_{i}s_{j}
    -h\sum_{i}s_{i},
\end{equation}
Where $\sum_{\left\langle i,j\right\rangle}$ means a summation over all pairs
of nearest neighbors. If we assume $J>0$, then the first term of the
Hamiltonian favors the alignment of the spins, while the second one 
favors the alignment with an external magnetic field.

The order parameter here is the spontaneous magnetization $M=\sum_i s_i / N$.
If the temperature is too the spins will tend to align themselves, giving rise
to an ordered state where $M>0$, while if the temperature is too high, thermal
fluctuations will misalign the spins and the resulting state is unordered
with $M=0$.

In his original workç~\cite{Ising1925}, Ising solved the model in one
dimension, showing that it does not display a phase transition at finite
temperature. From this result, he  incorrectly assumed the same would happen in
higher dimensions.

Figure~\ref{fig:ising_phase2} shows the phase space.

To this day no solution for three dimensions have been present, with some
authors arguing that this accomplishment is not even
possible~\cite{Istrail2000}.

Simulating the Ising model is particularly simple using the Metropolis-Hastings
algorithm~\cite{Hastings1970}. At each step chose a random spin of the lattice
and compute the change in energy $\Delta E$ that would occur if you were to
flip the spin. In a square lattice this is given by
\begin{equation}
    \Delta E=2s\left(i,j\right)\left[s\left(i+1,j\right)+
    s\left(i-1,j\right)+s\left(i,j+1\right)+s\left(i,j-1\right)\right],
\end{equation}
where $s(i,j)$ is the spin at position $x=ai$ and $y=aj$ with $a$ being the
lattice constant. You actually perform the flip $s(i,j)\rightarrow -s(i,j)$ if
$\Delta E < 0$. If not, you should still perform the flip randomly with
probability $\exp(-\Delta E / T)$. Figures~\ref{fig:ising}
and~\ref{fig:ising_phase}.

The Ising model is particularly famous for being the first 2D model of second
order phase transition to be analytically solved. You see, for most of the
first half of the 20th century, it was unclear whether the formalism developed
in statistical physics was enough to completely describe critical systems.
That's because the partition function, from which all statistical properties of
the system can be derived, seems to behave neatly as function of temperature.
It wasn't until the work of Onsager, a mathematical tour de force published in
1944~\cite{Onsager1944}, that it became clear that the singularities appear
only in the limit of infinite particles.


\begin{figure}
\begin{center}
    \includegraphics[width=\textwidth]{chapters/ch2-crit/figs/ising_phase2}
\end{center}
\caption{The phase space of the Ising model (left). The color represents the
    spontaneous magnetization $M$ as a function of the external magnetic field
    $h$ and temperature $T$. The black line represents the coexistence line
    where the phase transition is of first order, which means the magnetization
    changes discontinuously like it's shown in the blue line. In the critical
    point the change is continuous and a second order phase transition takes
    place.}
\label{fig:ising_phase2}
\end{figure}


\begin{figure}
\begin{center}
    \includegraphics[width=\textwidth]{chapters/ch2-crit/figs/ising}
\end{center}
\caption{Realizations of the Ising model with three different
    temperatures. The clusters of adjacent spin-up sites are colored according
    to how many sites belong to it. The subcritical regime is dominated by the
    large clusters. On the other hand, above the critical point, the system is
    dominated by thermal fluctuations, undermining cluster formation. At the
    critical point however, the clusters lack a characteristic length scale.
    One can observe that the image has a certain ``depth'' to it. This happens
    because clusters of all sizes are present, a mark of scale invariance,
    the most important property of critical systems.}
\label{fig:ising}
\end{figure}

\begin{figure}
\begin{center}
    \includegraphics[width=0.8\textwidth]{chapters/ch2-crit/figs/ising_phase}
\end{center}
\caption{Spontaneous magnetization as a function of temperature for the Ising
    model. The simulations were performed in a $128\times128$ square lattice.
    As temperature rises, thermal fluctuations dominate the spin dynamics
    destroying the correlations. Above the critical temperature of
    $T_c=2/\log(1+\sqrt{2})\approx 2.269$ the value of $M$ should reach zero,
    although due to finite size effects we still observe some magnetization
    beyond this point. The red line shows the illustrious solution
    developed by Onsager~\cite{Onsager1944}, where
    $M={[1-{(\sinh{2/T})}^{-4}]}^{1/8}$.}
\label{fig:ising_phase}
\end{figure}

\begin{figure}
\begin{center}
    \includegraphics[width=\textwidth]{chapters/ch2-crit/figs/ising_cx}
\end{center}
\caption{Lorem ipsum dolor sit amet, consectetur adipisicing elit, sed do
    eiusmod tempor incididunt ut labore et dolore magna aliqua. Ut enim ad
    minim veniam, quis nostrud exercitation ullamco laboris nisi ut aliquip ex
    ea commodo consequat. Duis aute irure dolor in reprehenderit in voluptate
    velit esse cillum dolore eu fugiat nulla pariatur. Excepteur sint occaecat
    cupidatat non proident, sunt in culpa qui officia deserunt mollit anim id
    est laborum.}
\label{fig:ising_cx}
\end{figure}



\subsection{Percolation}
\label{sec:perc}

While fiercely competing with the Ising model for the title of most popular
criticality model, percolation certainly wins in term of simplicity. Introduced
in 1957 by Broadbent and Hammersley~\cite{Broadbent1957}, the question they
were trying to answer was very humble: say we block a water pipe with a rock,
how porous should the rock be for the water to be able to flow, even if at a
very low flow rate? To find an answer, they modeled the rock as a square
lattice where each site can be either occupied by the rocky substrate or a
hole. The measure of the porosity is the ratio between the number of holes and
the total number of sites in the lattice. Of course, the more porous is a rock,
the more holes it has. Assuming these holes are randomly and uniformly
distributed on the substrate, they will often, by pure chance, fall into
clusters of neighboring holes, which are simply bigger holes. Water will be
able to flow through the rock only if we have a cluster that traverse the
system from one side to the other. This cluster is called the
\textit{percolating} (or spanning) cluster. So we can reframe the initial
question as follows: whats the smallest value of porosity where a percolating
cluster will always be present in the system. A 0\% porous rock would obviously
always be impermeable, and a 100\% porous rock is just no rock at all (how
zen!). But how about 50\%, or 40\%, or 60\%? The answer naturally depends on
the size of the lattice, but for a very large one we observe the existence of a
critical porosity $p_c$ below which we never observe a percolating cluster,
while above $p_c$ one is away present. See Figure~\ref{fig:isoperco} for some
realizations of the percolation process in each state: subcritical, critical
and supercritical.

This is the core of percolation theory, initially motivated by a transport
problem. From a more generic perspective, Stauffer summarizes it
better~\cite{Stauffer1994}: ``Each site of a lattice is occupied randomly with
probability $p$ independent of its neighbors. Percolation theory deals with the
clusters of neighboring occupied sites thus formed''. At this point it's clear
the system goes through a phase transition, each phase defined by the presence
or not of a spanning cluster, which promotes a global connectivity of the
system. We can study its properties by defining an order parameter $P$ as the
fraction of the system occupied by the largest cluster, that is $P=N_\ell/N$.
For small values of $p$, the largest cluster occupies a finite area of the
system, which is basically a null fraction for a large enough system. As $p$
approaches a critical probability $p_c$, the largest cluster grows to the order
of the system size. Figure~\ref{fig:isoperco2} shows how the order parameter
behaves for various sizes of a square lattice. It shows that as the system
reaches the thermodynamical limit, the order parameter closely behaves as a
second order phase transition.

Despite the initial motivation coming from transport theory, the percolation
model is much more general than that, it can be used to study.

The type of percolation described here, where we occupy the sites of a lattice
is called \textit{site percolation}. Alternatively we could instead occupy the
bonds. It is as if we had an empty isolating substrate with a potential
difference $V$ put along it. We then randomly insert resistors between the
sites with probability $p$. The system will be percolated when the it has a
bridge of resistors traversing it, along which electrical current can flow,
thus passing from an isolating phase to a conductive one.
Figure~\ref{fig:sitebond} illustrate the difference between site and bond
percolation.

The exact value of the critical point is non universal and depends on whether
it's site or bond percolation and on the lattice choice. The two more notorious
examples are square lattice site percolation, that has a critical point at
$p_c\approx0.592746$ and the triangular site percolation at $p_c=0.5$. Values
of the critical point for various other lattices are shown in
Table~\ref{tab:perc}.

The critical behavior of the percolation model is usually given by two critical
exponents. The first is the \textit{Fisher exponent} $\tau$, related to
the distribution of the cluster sizes at criticality
\begin{equation}
    \label{eq:ns1}
    n_s\sim s^{-\tau},\;\;\;\;\;\;p=p_c,
\end{equation}
where $s$ is the size of a cluster and $n_s$ is the number of clusters with
size $s$ in the system. Below the critical point, the distribution has an
exponential cutoff
\begin{equation}
    \label{eq:ns2}
    n_s\sim s^{-\tau}e^{-cs},\;\;\;\;\;\;p<p_c,
\end{equation}
where we define the other important critical exponent $\sigma$, related
to the scaling of the cutoff parameter $c$
\begin{equation}
    c\sim \left|p-p_c\right|^{-1/\sigma}.
\end{equation}
From these two exponents, $\tau$ and $\sigma$, we can determine all others.
For instance take the order parameter as defined before. From
Section~\ref{sec:universality} we know it scales as
\begin{equation}
    P\sim\left|p-p_c\right|^{-\beta}.
\end{equation}
We defined $P$ as the expected size of the largest cluster, which
we can assume should be distributed like $n_s(p_c)-n_s(p)$ for
$p < p_c$. Therefore, using Eqs.~\ref{eq:ns1} and~\ref{eq:ns2}
\begin{eqnarray}
    P & \sim & \int\left[n_{s}\left(p_{c}\right)-n_{s}\left(p\right)\right]sds\\
      & \sim & \int s^{1-\tau}\left[1-e^{-cs}\right]ds
\end{eqnarray}
Integrating by parts and making the transformation $s\rightarrow z/c$ yields
\begin{eqnarray}
P & \sim & c^{\tau-2}\int z^{2-\tau}e^{-z}dz\\
 & \propto & c^{\tau-2}\\
 & \propto & \left|p-p_{c}\right|^{\left(\tau-2\right)/\sigma}
\end{eqnarray}
which gives us the relation for $\beta$ as a function of $\tau$ and $\sigma$
\begin{equation}
    \beta=\frac{\tau-2}{\sigma}.
\end{equation}
    
In a similar way, we can find $\gamma$ by using the susceptibility, defined in
the percolation context as the second moment of the cluster size distribution
\begin{eqnarray}
    S & = & \int s^{2}n_{s}ds\\
      & = & \int s^{2-\tau}e^{-cs}ds\\
      & = & c^{\tau-3}\int z^{2-\tau}e^{-z}dz
    \label{eq:susperco}
\end{eqnarray}
Since the integral term in Eq.~\ref{eq:susperco} is a constant, we have
\begin{eqnarray}
 S & \propto & c^{\tau-3}\\
 & \propto & \left|p-p_{c}\right|^{-\left(3-\tau\right)/\sigma}.
\end{eqnarray}
which leads, following the definition of the $\gamma$ exponent
\begin{equation}
    \gamma=\frac{3-\tau}{\sigma}
\end{equation}

Once $\beta$ and $\gamma$ have been determined as a function of $\tau$ and
$\sigma$, all other critical exponents can be derived using
Eqs.~\ref{eq:scal1}--\ref{eq:scal2},~\ref{eq:hs1}, and~\ref{eq:hs2}. Here
are all of them (with $\beta$ and $\gamma$ repeated for completeness)
\begin{eqnarray}
    \alpha & = & 1-\frac{\tau-1}{\sigma}\\
    \beta  & = & \frac{\tau-2}{\sigma}\\
    \gamma & = & \frac{3-\tau}{\sigma}\\
    \delta & = & \frac{1}{\tau-2}\\
    \nu    & = & \frac{\tau-1}{\sigma d}\\
    \eta   & = & 2-d\frac{3-\tau}{\tau-1}
\end{eqnarray}

From a numerical standpoint, percolation does not pose too much problem, as
there are many algorithms around that allow for the simulation of very large
lattices, basically limited only by the memory of the
computer~\cite{Newman2000}.

The critical exponents of site percolation lattice have been exactly computed
in the triangular lattice using Schramm Loewner Evolutions~\cite{Smirnov2001}
(we'll get there in Chapter~\ref{ch:sle}). The values found are $\tau=187/91$
and $\sigma=36/91$. The other exponents can be found in Table~\ref{tab:univ}.

Just like the Ising model, the simplicity of percolation is ripe for the
development of derivative models, aiming to elucidate other aspects of phase
transitions. These include cooperative sequential adsorption~\cite{Araujo2013},
explosive percolation~\cite{Achlioptas2009} and invasive
percolation~\cite{Wilkinson1983}, to name just a few.

\begin{figure}
\begin{center}
    \includegraphics[width=\textwidth]{chapters/ch2-crit/figs/isoperco}
\end{center}
\caption{Realizations of the percolation model in a square lattice with three
    different occupation probabilities $p$. Black sites are unoccupied, blue
    ones are occupied, and the largest cluster is painted yellow. For small
    values of $p$, there is no cluster that connects opposite sides of the
    systems. Above the critical point however, the largest cluster promotes a
    global connectivity, or, in terms of transport, the system becomes
    permeable.}
\label{fig:isoperco}
\end{figure}

\begin{figure}
\begin{center}
    \includegraphics[width=0.7\textwidth]{chapters/ch2-crit/figs/isoperco2}
\end{center}
\caption{The order parameter of the percolation model as a function of the
    occupation probability for various system sizes. The order parameter here
    is defined as the fraction of the system occupied by the largest cluster.
    In the thermodynamical limit, the largest cluster have a finite size for
    $T<T_c\approx 0.592746$, that is, it occupies a negligible fraction of the
    system. Above the critical point the largest cluster is infinite and occupies
    a finite fraction.}
\label{fig:isoperco2}
\end{figure}

\begin{figure}
\begin{center}
    \includegraphics[width=0.8\textwidth]{chapters/ch2-crit/figs/sitebond}
\end{center}
\caption{Examples of site and bond percolation in a square lattice.
    In site percolation each site is occupied with probability $p$, while
    in bond percolation any two neighboring sites are connected with a bond
    with probability $p$.}
\label{fig:sitebond}
\end{figure}


\begin{table}[t]
\newcolumntype{L}[1]{>{\raggedright\let\newline\\\arraybackslash\hspace{0pt}}m{#1}}
\begin{centering}
\begin{tabular}{lll}
\bottomrule[0.1mm]
\toprule[0.1mm]
\textbf{Lattice} & \textbf{Site} & \textbf{Bond}     \\
\toprule[0.1mm]
Square           & 0.592746      & 0.5      \\[0.2cm]
Triangular       & 0.5           & 0.5      \\[0.2cm]
Honeycomb        & 0.6962        & 0.65271  \\[0.2cm]
Diamond          & 0.6962        & 0.65271  \\[0.2cm]
Simple Cubic     & 0.3116        & 0.2488   \\[0.2cm]
BCC              & 0.246         & 0.1803   \\[0.2cm]
FCC              & 0.198         & 0.119    \\[0.2cm]
\toprule[0.1mm]
\toprule[0.1mm]
\end{tabular}
\par\end{centering}
\caption{Percolation threshold for site and bond percolation in several 2D and
    3D regular lattices. Reproduced from~\cite{Stauffer1994}.}
\label{tab:perc}
\end{table}

