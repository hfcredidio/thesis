\chapter{Phase Transitions and Critical Phenomena}
\label{ch2-crit}

%sources
    %Henkel - Conformal Invariance and Critical Phenomena - Ch1
    %Gitterman - Ch1
    %Nishimore - Ch1
    %Sole - Ch1

    %DIFFUSION-LIMITED GROWTH IN BACTERIAL COLONY FORMATION Mitsugu MATSUSHITA a
    %and Hiroshi FUJIKAWA b (LDA FIG)
    %http://people.umass.edu/machta/images/dla.html

A big part of the physics effort is to understand how the different components
of a system interact with one another. The actual identity of these components
depends on the context, they can be elementary particles when studying high
energy physics, atoms and molecules when studying the electronic properties of
a material, living cells when observing the growing pattern of a colony of
bacteria, or even people when trying to design buildings with safe exit routes.
More often than not, these objects interact in a very complicated manner and a
lot of effort have been put in creating realistic models to capture the
essential physical properties of these systems. Some phenomena however, have
properties that seem to depend not on the details of the interactions, but on
the ways the parts of a system can organize itself in the border of phase
transition.

Let's take a step back and look again at the components that make up a physical
system. Usually they can take many different patterns of organization, and each
of these patterns can behave in wildly different ways despite having the same
composition. Liquid water and ice, for instance, are the exact same substance,
H${}_2$O, the only difference being how orderly the molecules find themselves.
When in its solid form, the cohesive forces between the water molecules
dominate the dynamics because they do not have enough kinetic energy to move
around; they are locked in place. If the temperature is sufficiently high they
get excited enough to be able move around between themselves, thus passing to
the liquid state. Raise the temperature even more and the thermal movement
starts overshadowing the attractive forces and the molecules can now move
freely without even noticing the one another (unless, that is, if they directly
collide). That is the gaseous state. The arrangement of water molecules moves
from a more orderly state (solid) to a more disordered one (gaseous) as the
temperature rises. We call each of these organization states a \textit{phase},
and the phase in which a system is found is usually dependent on an external
parameter which in the case of water is temperature.

In order to better observe the different phases of a system we make use of a
phase diagram. It shows the particular phase a system is found for every value
of the control parameter. Figure~\ref{fig:water}a shows the phase diagram
of water.

%Let's take a step back and look again at the components that make up a physical
%system. Usually they can take many different patterns of organization, which
%depend on the adjustment of some external parameter, like energy density,
%temperature, nutrient concentration, or whether or not there is a fire going on
%on the building. We call these arrangements \textit{phases}, and the system can
%behave in wildly different ways in different phases despite having the same
%composition. Examples are abundant, and anyone who has been through elementary
%school is familiar with the three ``standard'' phases of matter: solid, liquid
%and gas. The control parameter here is the temperature of a substance. When it
%is low enough, the cohesive forces between molecules dominates the dynamics,
%locking them in place and giving rise to the solid state. When the temperature
%is too high, thermal movement dominates and the molecules are basically free to
%move around. The liquid phase emerges in an intermediary state, when the
%particles have too much energy to remain fixed but not enough to move
%unimpeded. Figure~\ref{fig:water}a shows the very famous phase diagram of
%water. It shows in which phase the system is found for every value of pressure
%and temperature. The black lines represent the border between each phase, when
%the control parameters of the system change in a way that crosses the boundary
%of a phase, the system undergoes a \textit{phase transition}.

To properly quantify the phenomenon of phase transitions, it is necessary to
measure how the components of the systems are organized. This quantity, which
is defined differently for each system, is called the \textit{order parameter}
and is a function of the control parameters. In the case of water we can take
the density as an order parameter. As can be seen in Figure~\ref{fig:water}b,
when a body of water undergoes a phase transition, there is a drastic jump in
density. Other thermodynamic potentials display a similar behavior, but of
particular interest if the \textit{free energy} $F=E-TS$ because it captures
the dispute between order (represented by $E$, the energy) and disorder
(represented by $S$ the entropy) typical of phase transitions. The usual type
of transition that takes place in water when the system cross a phase border is
called a \textit{first order phase transition}.

One might notice however that the liquid-gas critical line does not extend all
the way through the phase diagram. In fact it ends in a point labeled
\textit{critical point}. At this point a very peculiar phenomenon takes place.
Instead of taking a jump, the density varies continuously from one phase to the
other. At the exact point where the two phases meet they become
indistinguishable from one another. This phase transition is radically different
from the others described before, because they present a series of

Below we present two important models in the study of phase transition
Figure~\ref{fig:water} 

\begin{figure}[h]
\begin{center}
    \includegraphics[scale=1.0]{chapters/ch2-crit/figs/water}
\end{center}
\caption{Phase diagram of water (a). Here, the three usual phases are
    distinguished, each separated from the other by a critical line. The phase
    transition that happens when the system traverse a line is characterized by
    a discontinuous jump in the density, as shown in (b) for the liquid-gas
    transition at $P=1$ atm. For $P=217.7$ atm however the same transition is
    continuous, as shown in (c). In the vicinity of this phase transition, at
    $T\approx374.4^\circ$C, the two phases become indistinguishable, and
    display a number of peculiarities. Systems in such state are called
    critical systems. Reproduced from [???].}
% taken from Sole - page 7
\label{fig:water}
\end{figure}


\section{Ising Model}
\label{sec:ising}

The Ising model represents the golden standard of phase transition models
because it unites a quite simple physical description with the fact it can be
exactly solved in both one and two dimensions, a claim very few models can
make.

The system is composed of a number of classical spins $\{s_i\}$ that can take
one of two values $1$ or $-1$. The are arranged in a lattice and are allowed to
interact with its first nearest neighbors. The Hamiltonian of the systems is
given by
\begin{equation}
    \label{eq:ising}
    H\left(\left\{s_{i}\right\}\right) = 
        -J\sum_{\left\langle i,j\right\rangle}s_{i}s_{j}
        -h\sum_{i}s_{i}.
\end{equation}
Where $\sum_{\left\langle i,j\right\rangle}$ means a summation over all pairs
of nearest neighbors. If we assume $J>0$, then the first term of the
Hamiltonian favors the alignment of the spins, while the second one 
favors the alignment with an external magnetic field.

Just like the water example, the phase diagram of the Ising model is dependent
on two variables: the external magnetic field $h$ and the temperature $T$.

\begin{figure}
\begin{center}
    \includegraphics[scale=0.4]{chapters/ch2-crit/figs/ising_phase2}
\end{center}
\caption{The phase space of the Ising model (left). The color represents the
    spontaneous magnetization $M$ as a function of the external magnetic field
    $h$ and temperature $T$. The black line represents the critical line where
    the phase transition is of first order, which means the magnetization
    changes discontinuously like it's shown in the blue line. In the critical
    point the change is continuous and a second order phase transition takes
    place. Above the critical point no phase transition takes place.}
\label{fig:ising_phase2}
\end{figure}


\begin{figure}
\begin{center}
    \includegraphics[scale=0.4]{chapters/ch2-crit/figs/ising}
\end{center}
\caption{Realizations of the Ising model with three different
    temperatures. The clusters of adjacent spin-up sites are colored according
    to how many sites belong to it. The subcritical regime is dominated by the
    large clusters. On the other hand, above the critical point, the system is
    dominated by thermal fluctuations, undermining cluster formation. At the
    critical point however, the clusters lack a characteristic length scale.
    One can observe that the image has a certain ``depth'' to it. This happens
    because clusters of all sizes are present, a mark of scale invariance,
    the most important property of critical systems.}
\label{fig:ising}
\end{figure}

\begin{figure}
\begin{center}
    \includegraphics[scale=0.4]{chapters/ch2-crit/figs/ising_phase}
\end{center}
\caption{Spontaneous magnetization as a function of temperature for the Ising
    model. The simulations were performed in a $128\times128$ square lattice.
    As temperature rises, thermal fluctuations dominate the spin dynamics
    destroying the correlations. Above the critical temperature of
    $T_c=2/\log(1+\sqrt{2})\approx 2.269$ the value of $M$ should reaches zero,
    although due to finite size effects we still observe some magnetization
    beyond this point. The red line shows the illustrious solution
    developed by Onsager, where $M={[1-{(\sinh{2/T})}^{-4}]}^{1/8}$.}
\label{fig:ising_phase}
\end{figure}


\section{Percolation}
\label{sec:perc}

While the Ising model certainly wins in terms of popularity, very few models
match the simplicity of percolation. Introduced in 1957 by Broadbent and
Hammersley, the passing decades saw its rise in prominence due to it's high
applicability ranging from transport in porous media to the propagation of
infectious diseases, with pretty much everything in between.

The original question Broadbent and Hammersley posed when coming up with
percolation was as fallows: if we submerge a large porous rock in water, will
the water penetrate the rock all the way to its center? In order to answer this
question they proposed the following model. We take a square lattice in any
number of dimensions wanted; this will be our rock.

\begin{figure}
\begin{center}
    \includegraphics[scale=0.4]{chapters/ch2-crit/figs/isoperco}
\end{center}
\caption{Realizations of the percolation model in a square lattice with three
    different occupation probabilities $p$. Black sites are unoccupied, blue
    ones are occupied, and the largest cluster is painted yellow. For small
    values of $p$, there is no cluster that connects opposite sides of the
    systems. Above the critical point however, the largest cluster promotes a
    global connectivity, or, in terms of transport, the system becomes
    permeable.}
\label{fig:isoperco}
\end{figure}

\begin{figure}
\begin{center}
    \includegraphics[scale=0.4]{chapters/ch2-crit/figs/isoperco2}
\end{center}
\caption{The order parameter of the percolation model as a function of the
    occupation probability for various system sizes. The order parameter here
    is defined as the fraction of the system occupied by the largest cluster.
    In the thermodynamical limit, the largest cluster have a finite size for
    $T<T_c\approx 0.592746$, that is, it occupies a negligible fraction of the
    system. Above the critical point the largest cluster is infinite and occupies
    a finite fraction.}
\label{fig:isoperco2}
\end{figure}

%\begin{figure}
%\begin{center}
    %\includegraphics[scale=0.6]{chapters/ch2-crit/figs/bacteria}
%\end{center}
%\caption{Images taken from [???] and [???].}
%\label{fig:bacteria}
%\end{figure}
