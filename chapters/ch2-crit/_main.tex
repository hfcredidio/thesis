\chapter{Phase Transitions and Critical Systems Redux}
\label{ch:criticality}


We live in a world where particle physics can reach the headlines. This
complicated kind of science can sound esoteric to people. It may come as a
surprise that some of the most puzzling physics problems fit in a checkerboard.

Let's make a bet. I'll take a checkerboard and throw a coin. If the coin falls
on heads I'll put a checkers piece in the first tile of the board, if it falls
on tails I'll leave it empty. The process is repeated for every tile of board
successively until we end up with something the like board shown in Fig.~??.
Naturally the pieces are randomly distributed along the board, but due to
random chance some of them form dense clusters of neighboring pieces. We
consider two pieces as neighbors if are in adjacent tiles along the vertical or
horizontal directions, but not in the diagonal. Here are the rules of the bet.
If by the end of the coin-throwing process you can trace a path from the bottom
of the board all the way through the top passing only through neighboring
pieces you get, say, double your money. In other words, you win if it exists a
continuous cluster that connect the bottom and the top row of the board. If
there's no such cluster you lose all you've bet. There's a twist though, we'll
be playing in a gigantic $1,000,000$ by $1,000,000$ tiles checkerboard, instead of
the more traditional 8 by 8. How much money would you be willing to bet in this
game?

To answer this question let's take a step back and imagine we're playing the
game on a board with only a single tile. In this case your probability of
winning is 50\%, because we throw the coin only once, and filling the single
tile will connect both sides of the board.

On a 2 by 2 board there are 16 possible ways in which we can lay down the
pieces. Of these, only 7 result in a configuration in which you win the bet.
This can be seen in Fig.~??. The probability of winning is therefore 7/16 or
approximately 44\%. Following the same line of thought, the probability of
winning on a 3 by 3 board is around 38\%, and 34\% for a 4 by 4. For larger
sizes this analysis gets unwieldy due to the number of possible configurations
of an L by L board being $2^{L^2}$, which means a standard 8 by 8 board have
more configurations than there are stars in the Milky Way by several orders of
magnitude. We can compute the probability of winning approximately by
simulating a game many times and simply counting the fraction of times the game
resulted in victory. The result is shown in Fig.??. As you can see, the
probability of winning falls steadily as the size of the board grows. But will
this behavior continue until the probability reaches zero or will the curve
flatten and eventually settle for a probability larger that zero? Running the
same simulation for a 1,000 by 1,000 board (which has a number of
configurations that completely dwarfs the number of particles in the know
universe by several thousand orders of magnitude) for 10,000 times yielded not
a single victory. Thus, It is not far fetched to imagine the curve int Fig.??
will indeed approach zero.
% TODO: Rewrite folowing sentence, it's shitty the way it is.
This happens because all configurations are equally probable, because every
coin toss is independent and perfectly fair. When we grow the size of the
board, the number of winning configurations diminishes as a fraction of all
possible configurations, this rendering the probability of winning.

Going back to our bet, how much would you be willing to bet on a 1,000,000 by
1,000,000 board? The only reasonable answer is none. Simply playing the game 
is an assured way of losing all of your money.

% TODO: Explain biased coin

So if you have access to such a biased coin and possess the appropriate moral
inclination to use it, the only logical action would be to bet all of your
money in the game, as victory is a statistical certainty. 

I propose now another another game, more like a riddle really. I show you
various coins, each coin has distinct number between 0.5 and 0.7 written that
indicates the bias of the coin. There are coins that yield heads 53\% of the
time, other with 68.97\% and so on. You should chose a coin and we'll play the
game as described before. If you lose the game you lose the money as usual, but
if you win the game you will get your money divided by the bias of the coin you
chose. The smaller the value of the coin, the larger will be your prize. We
know that choosing a coin with $p=0.5$ is silly because you will never win
the game with it. Choosing the coin with $p=0.7$ yields a certain victory but
with the smallest prize. What coin should you chose to certainly win, but
that yields you the best prize, that is, what is the smallest bias that a
coin must have to guarantee that you will win the game?

The answer is $p=0.5962...$. This number marks a transition between two states
of the game.

One would imagine that any coin that has a bias larger than $0.5$ would result
in certain victory, given that this coin would skew the distribution towards
configurations with more occupied sites. While this line of thinking is sound,
it is not correct, a bias of, say $0.55$ is not enough to ensure a victory
in our oversized board.
