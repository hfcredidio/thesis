\chapter{Schramm-Loewner Evolutions of Strongly Anisotropic Systems}
\label{ch:res}

In this work we take the effort to relate the Stochastic Loewner Evolution with
strongly anisotropic systems. Because such systems are not conformally
invariant, the driving process of these models must not be a simple Brownian
motion.

\section{Lévy-Flights and Schramm-Loewner Evolutions}

One possible source of anisotropy in the context of SLE is the presence of
anomalous diffusion in the driving process. A stochastic process $X_t$ is
said to display anomalous diffusion if the mean squared displacement
behaves asymptotically as
\begin{equation}
    \left\langle X_t^2 \right\rangle \sim bt^\alpha
\end{equation}
with $\alpha\neq 1$. Processes that have $\alpha < 1$ are called subdiffusive
models and those with $\alpha > 1$ are called superdiffusive.

There are several models that present anomalous diffusion. Rushkin et Al [..]
proposed using L\'evy-flights as a driving process. L\'evy-flights are defined as
processes that have jump size distributed as a power law, that is
\begin{equation}
    P(X_{t+dt} - X_t \in [x, x+dx]) = \frac{C}{|x|^{1+\mu}}dxdt
\end{equation}
where $C$ is a normalization constant.

