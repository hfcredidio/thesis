\chapter{The \texttt{loew} Python Package}
\label{ch:py}

A subproduct of the numerical work presented in Chapter~\ref{ch:asle} was the
creation of several routines for simulation of Loewner evolution traces and
driving functions. We decided to pack these routines into a lightweight and
efficient library for the Python programming language.

Python is a general purpose high-level programming language created by Guido
van Rossum~\cite{vanRossum1995} in the early 90's aiming for a simpler and more
expressive language. In the last decade, a series of libraries placed Python as
a viable alternative to other programming languages normally used in science,
like MATLAB and R. These libraries include Numpy, Scipy, matplotlib, among many
others. The wide range of scientific and non-scientific packages allied with
its simplicity and stellar community support, allowed Python to gain a lot of
space in the scientific community. Even one of the most cited drawbacks of the
language, its lack of performance compared to compiled languages like C and
Fortan, has been addressed by proejcts like PyPy, Numba and Cython.
