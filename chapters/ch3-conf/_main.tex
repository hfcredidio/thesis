\chapter{Conformal Invariance and Critical Phenomena}
\label{ch:conf}

\newcommand{\G}{
    \left\langle
        \phi_{1}\left(z_{1},\bar{z}_{1}\right)
        \phi_{2}\left(z_{2},\bar{z}_{2}\right)\ldots
    \right\rangle
}


\begin{eqnarray*}
    \phi\left(z'\right) & = & \phi\left(z\right)+
                              \epsilon\left(z\right)\partial\phi\left(z\right)\\
                        & = & \phi\left(z\right)+
                              \sum_{n=-\infty}^{\infty}\epsilon_{n}z^{n+1}
                              \partial_{z}\phi\left(z\right)\\
                        & = & \phi\left(z\right)-
                              \sum_{n=-\infty}^{\infty}\epsilon_{n}\ell_{n}
                              \phi\left(z\right)
\end{eqnarray*}
Where we define the generator of holomorphic functions
\begin{align*}
    \ell_{n}=-z^{n+1}\partial_{z} \\
    \bar{\ell}_{n}=-\bar{z}^{n+1}\partial_{\bar{z}}
\end{align*}
These operators form an infinite dimensional algebra called \textit{loop
algebra}, where one can easily demonstrate the commutation relations
\begin{align}
    \left[\ell_{n},\ell_{m}\right]&=(m-n)\ell_{n+m}\\
    \left[\bar{\ell}_{n},\bar{\ell}_{m}\right]&=(m-n)\bar{\ell}_{n+m}\\
    \left[\ell_{n},\bar{\ell}_{m}\right]&=0
\end{align}

There is a closed subalgebra for $n\in\{-1, 0, 1\}$. This can be easily checked
by noticing that $[\ell_0,\ell_{\pm 1}] = \pm\ell_{\pm 1}$ and $[\ell_1,
\ell_{-1}] = -2\ell_0$. This subalgebra is the generator of \textit{projective
conformal transformations}. One can prove that in $d\ge3$ this subalgebra
is the whole group of conformal transformations.

We now consider fields that transform covariantly according to projective
conformal transformations, that is

\begin{equation}
    \phi\left(f\left(z\right),\bar{f}\left(\bar{z}\right)\right)=
    (\partial_{z}f)^{h}(\partial_{\bar{z}}\bar{f})^{\bar{h}}\phi\left(z,\bar{z}\right)
\end{equation}
Fields that obey this relation are called \textit{quasi-primary fields}.
The exponents $h$ and $\bar{h}$ are called \textit{conformal weights}.
The correlation functions therefore obey the relation
\begin{equation}
    \left\langle
        \phi_{1}\left(z_{1},\bar{z}_{1}\right)
        \phi_{2}\left(z_{2},\bar{z}_{2}\right)\ldots
    \right\rangle =
    \left(
        \prod_{i}\left( \partial_{z_{i}}f_{i}\right)^{h_{i}}
        \left(\bar{\partial}_{\bar{z}_{i}}\bar{f}_{i}\right)^{\bar{h}_{i}}
    \right)
    \left\langle
        \phi'_{1}\left(f_{1},\bar{f}_{1}\right)
        \phi'_{2}\left(f_{2},\bar{f}_{2}\right)\ldots
    \right\rangle 
\end{equation}
In the case of a change of scale $f(z)=z/b$ and $\bar{f}(\bar{z})=\bar{z}/b$
with $b$ real we have
\begin{equation}
    \phi'_{j}\left(f,\bar{f}\right)=
    b^{h_{j}+\bar{h}_{j}}\phi\left(z,\bar{z}\right)
\end{equation}
In the case of a rotation $f(z)=e^{i\theta z}$ and
$\bar{f}(\bar{z})=e^{-i\theta \bar{z}}$, which leads to
\begin{equation}
    \phi'_{j}\left(f,\bar{f}\right)=
    e^{-i\left(h_{j}-\bar{h}_{j}\right)\theta}\phi\left(z,\bar{z}\right).
\end{equation}
These relations allow us to identify the conformal weights with the
scaling dimensions of the field $x_i=h_i+\bar{h}_i$ and
their spin $s_i=h_i-\bar{h}_i$.

Suppose now we apply an infinitesimal conformal map $z=z+\epsilon(z)$
and $\bar{z}=\bar{z}+\epsilon(\bar{z})$.
We then have
\begin{equation}
    \delta_{\epsilon\bar{\epsilon}}\left(\phi\right)=
    \left(
        \epsilon\partial+\bar{\epsilon}\bar{\partial}+
        h\partial\epsilon+\bar{h}\bar{\partial}\bar{\epsilon}
    \right)\phi
\end{equation}
We take the condition of conformal invariance that
the n-point correlation function is invariant to conformal
transformations, that is
\begin{multline}
    \delta_{\epsilon\bar{\epsilon}} \G = 0 \\
    = \sum_{i}\left(
        \epsilon_{i}\partial_{i}+
        \bar{\epsilon}_{i}\bar{\partial}_{i}+
        h_{i}\partial_{i}\epsilon_{i}+
        \bar{h}_{i}\bar{\partial}_{i}\bar{\epsilon}_{i}
    \right) \G
\end{multline}

Taking only the elements -1, 0 and 1
\begin{eqnarray}
    \sum_{i}\partial_{i} \G& = & 0 \\
    \sum_{i}\left(z_{i}\partial_{i}+h_{i}\right) \G& = & 0 \\
    \sum_{i}\left(z_{i}^{2}\partial_{i}+2h_{i}z_{i}\right) \G& = & 0
\end{eqnarray}
which are called \textit{projective Ward identities}.
