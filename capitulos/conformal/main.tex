\chapter{Conformal Invariance and Critical Phenomena}
\label{ch:conf}

\begin{eqnarray*}
    \phi\left(z'\right) & = & \phi\left(z\right)+
                              \epsilon\left(z\right)\partial\phi\left(z\right)\\
                        & = & \phi\left(z\right)+
                              \sum_{n=-\infty}^{\infty}\epsilon_{n}z^{n+1}
                              \partial_{z}\phi\left(z\right)\\
                        & = & \phi\left(z\right)-
                              \sum_{n=-\infty}^{\infty}\epsilon_{n}\ell_{n}
                              \phi\left(z\right)
\end{eqnarray*}
Where we define the generator of holomorphic functions
\begin{align*}
    \ell_{n}=-z^{n+1}\partial_{z} \\
    \bar{\ell}_{n}=-\bar{z}^{n+1}\partial_{\bar{z}}
\end{align*}
These operators form an infinite dimensional algebra called \textit{loop
algebra}, where one can easily demonstrate the commutation relations
\begin{align}
    \left[\ell_{n},\ell_{m}\right]&=(m-n)\ell_{n+m}\\
    \left[\bar{\ell}_{n},\bar{\ell}_{m}\right]&=(m-n)\bar{\ell}_{n+m}\\
    \left[\ell_{n},\bar{\ell}_{m}\right]&=0
\end{align}

There is a closed subalgebra for $n\in\{-1, 0, 1\}$. This can be easily checked
by noticing that $[\ell_0,\ell_{\pm 1}] = \pm\ell_{\pm 1}$ and $[\ell_1,
\ell_{-1}] = -2\ell_0$. This subalgebra is the generator of \textit{projective
conformal transformations}. One can prove that in $d\ge3$ this subalgebra
is the whole group of conformal transformations.

We now consider fields that transform covariantly according to projective
conformal transformations, that is

\begin{equation}
    \phi\left(f\left(z\right),\bar{f}\left(\bar{z}\right)\right)=
    (\partial_{z}f)^{h}(\partial_{\bar{z}}\bar{f})^{\bar{h}}\phi\left(z,\bar{z}\right)
\end{equation}

Fields that obey this relation are called \textit{quasi-primary fields}.
