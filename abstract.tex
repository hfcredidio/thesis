\setlength{\absparsep}{18pt} % ajusta o espaçamento dos parágrafos do resumo
\begin{resumo}
    In this work, we aim to explore two areas of interest in complexity science: visual
    search and Schramm-Loewner evolution.  In the first one we aim to study how
    visual search strategies evolve with increasing complexity of the visual
    scene. We observe the qualitative evolution from systematic search
    strategies to random ones, and aim to establish an appropriate
    quantitative measure of this transition in behavior.  In the second work we
    set to study a model for conformally invariant critical systems called
    Schramm-Loewner evolution (SLE). In particular we aim to determine if
    several lattice models can be modeled as SLE in the continuum limit. We do
    this by numerically analyzing the winding angle, left passage probability
    and driving function, comparing with SLE curves with same fractal dimension.
    \vspace{\onelineskip}
    \noindent 

    \textbf{Keywords}: sle. criticality.
\end{resumo}

% resumo em portugues
\begin{resumo}[Resumo]
\begin{otherlanguage*}{brazil}
    Neste trabalho, nos propomos a explorar duas áreas de interesse para a ciência
    da complexidade: buscas visuais e evoluções de Schramm-Loewner. No primeiro,
    nosso objetivo é estudar como as estratégias de busca visual evoluem com
    o aumento da complexidade da cena visual. Nós observamos 
    qualitativamente uma evolução de estratégias sistemáticas em direção a
    estratégias aleatórias, e pretendemos estabelecer uma medida quantitativa
    para essa transição de comportamento. No segundo trabalho queremos estudar
    evolução de Schramm-Loewner (SLE), um modelo de sistemas criticos com invariância
    conforme. Em particular, pretendemos determinar se uma série de modelos em
    redes regulares podem ser modelados com SLE no limite do contínuo. Fazemos isso
    analizando numericamente os ângulos de enrolamento, probabilidade de passagem
    à esquerda e função de direcionamento, comparando com curvas SLE de mesmas
    dimensão fractal.
    \vspace{\onelineskip}
    \noindent 

    \textbf{Palavras-chaves}: sle. criticalidade.
\end{otherlanguage*}
\end{resumo}
