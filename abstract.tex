\setlength{\absparsep}{18pt} % ajusta o espaçamento dos parágrafos do resumo
\begin{resumo}
    We disclose the origin of anisotropic percolation perimeters in terms of
    the Stochastic Loewner Evolution (SLE) process. Precisely, our results from
    extensive numerical simulations indicate that the perimeters of
    multi-layered and directed percolation clusters at criticality are the
    scaling limits of the Loewner evolution of an anomalous Brownian motion,
    being superdiffusive and subdiffusive, respectively. The connection between
    anomalous diffusion and fractal anisotropy is further tested by using
    long-range power-law correlated time series (fractional Brownian motion) as
    driving functions in the evolution process. The fact that the resulting
    traces are distinctively anisotropic corroborates our hypothesis. Under the
    conceptual framework of SLE, our study therefore reveals new perspectives
    for mathematical and physical interpretations of non-Markovian processes in
    terms of anisotropic paths at criticality and vice-versa.
    \vspace{\onelineskip}
    \noindent 

    \textbf{Keywords}: sle\. criticality.
\end{resumo}

% resumo em portugues
\begin{resumo}[Resumo]
\begin{otherlanguage*}{brazil}
    Lorem ipsum dolor sit amet, consectetur adipisicing elit, sed do eiusmod
    tempor incididunt ut labore et dolore magna aliqua. Ut enim ad minim
    veniam, quis nostrud exercitation ullamco laboris nisi ut aliquip ex ea
    commodo consequat. Duis aute irure dolor in reprehenderit in voluptate
    velit esse cillum dolore eu fugiat nulla pariatur. Excepteur sint occaecat
    cupidatat non proident, sunt in culpa qui officia deserunt mollit anim id
    est laborum.
    \vspace{\onelineskip}
    \noindent 

    \textbf{Palavras-chaves}: sle\. criticalidade.
\end{otherlanguage*}
\end{resumo}
