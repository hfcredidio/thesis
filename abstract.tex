\setlength{\absparsep}{18pt} % ajusta o espaçamento dos parágrafos do resumo
\begin{resumo}
    We disclose the origin of anisotropic percolation perimeters in terms of
    the Stochastic Loewner Evolution (SLE) process. Precisely, our results from
    extensive numerical simulations indicate that the perimeters of
    multi-layered and directed percolation clusters at criticality have as
    scaling limits the Loewner evolution of an anomalous Brownian motion,
    being superdiffusive and subdiffusive, respectively. The connection between
    anomalous diffusion and fractal anisotropy is further tested by using
    long-range power-law correlated time series (fractional Brownian motion) as
    driving functions in the evolution process. The fact that the resulting
    traces are distinctively anisotropic corroborates our hypothesis. Under the
    conceptual framework of SLE, our study therefore reveals new perspectives
    for mathematical and physical interpretations of non-Markovian processes in
    terms of anisotropic paths at criticality and vice-versa.
    \vspace{\onelineskip}
    \noindent 

    \textbf{Keywords}: SLE\@. Criticality. Percolation. Anisotropy.
\end{resumo}

% resumo em portugues
\begin{resumo}[Resumo]
\begin{otherlanguage*}{brazil}
    Usamos Evoluções de Schramm-Loewner (SLE) para expor a origem de interfaces
    anisotrópicas presentes em percolação. Mais precisamente, nossos
    resultados, obtidos através de extensas simulações numéricas, indicam que
    os perímetros de agregados encontrados em duas variantes do modelo de
    percolação têm como limite termodinâmico evoluções de Loewner dirigidas por
    movimentos Brownianos anômalos. Percolação em multi-camadas exibe
    comportamento superdifusivo e percolação direcionada subdifusivo.
    Testamos a conexão entre difusão anômala e anisotropia usando séries
    temporais com correlação de longo alcance em lei de potência (movimentos
    Brownianos fracionários) como funções diretoras nas SLE\@. Nossa hipótese
    é corroborada pelo fato de que os traços obtidos são distintamente
    anisotrópicos. Sob a estrutura conceitual das SLE, nosso estudo revela
    novas perspectivas para interpretações matemáticas e físicas de processos
    não-Markovianos em termos de caminhos anisotrópicos em criticalidade, e
    vice-versa.
    \vspace{\onelineskip}
    \noindent 

    \textbf{Palavras-chaves}: SLE\@. Criticalidade. Percolação. Anisotropia.
\end{otherlanguage*}
\end{resumo}
